\documentclass[12pt]{amsart}
\usepackage{mathptmx}
\usepackage{fullpage, hyphenat}
\addtolength{\footskip}{1cm}

\newcommand{\definition}{\ \\ \noindent \textbf{Def.} }
\newcommand{\note}{\ \\ \noindent  \textbf{Note:} }
\newcommand{\theorem}{\ \\ \noindent  \textbf{Theorem.} }
\newcommand{\lemma}{\ \\ \noindent  \textbf{Lemma.} }
\newcommand{\remark}{\ \\ \noindent \textbf{Remark.} }
\newcommand{\notation}{\ \\ \noindent \textbf{Notation.} }
\newcommand{\corollary}{\ \\ \noindent \textbf{Cor.} }
\newcommand{\question}{\ \\ \noindent \textbf{Question.} }
\newcommand{\pf}{\ \\ \noindent \textbf{Proof.} }
\newcommand{\lemmaone}{\ \\ \noindent  \textbf{Lemma 1.} }
\newcommand{\lemmatwo}{\ \\ \noindent  \textbf{Lemma 1.} }


\begin{document}
\begin{center}
{\Large HEEGAARD DOCUMENTATION}

by John Berge\footnote{Translated into \LaTeX\ by Nathan Dunfield.
  Hopefully no errors were introduced in this process}
\end{center}

                                                                                
\section{INTRODUCTION}
                
                
This file includes some detailed information about the mathematical
underpinnings of the program and also takes a closer look at many of
the program's routines.

The design decision to have the program use presentations of compact
orientable 3-manifolds as its basic data, and the subsequent need for
the program to be able to efficiently solve``realization" problems in
order to recover Heegaard diagrams of the 3-manifolds involved, turns
out to lead to a surprisingly complex program. In this file, we try to
indicate what some of the problems are which the program faces, and to
also indicate what the program's current strategies are for dealing
with them.

Ultimately, the goal is to make understanding compact orientable
3-manifolds easier, and to make Heegaard diagrams as useful a tool in
that understanding as possible. The program attempts to take a step in
that direction, but it should be thought of as a provisional step,
with the program itself, its approach and its routines all subject to
revision and improvement. We wish to also mention that we do not
believe that the road to understanding compact orientable 3-manifolds
must lead through Heegaard diagrams.  Rather, the idea is only to make
such diagrams as useful as they can be. In this respect, it is
probably worth stressing, that once the program has determined that a
presentation is uniquely realizable, then the program``knows" the
3-manifold, and, in principle, additional routines could be added to
the program to compute any of one's favorite 3-manifold invariants, or
routines could be added to search for``interesting" features of the
3-manifold.

Finally, to facilitate making modifications, improvements, and
additions to the program, we will make the source code available to
those who wish to proceed further.



\section{SOME USEFUL DEFINITIONS, NOTATION, AND GENERAL INFORMATION.}
                                                
                                                                                
\definition     Let $P$ be a finite presentation on a set of $g$ generators with $n$ relators $R_i$,
        $1 \leq i \leq n$, such that $R_i$ is freely and cyclically reduced for $1 \leq i \leq n$. We say $P$ is
        ``realizable" if there exists a handlebody $H$, of genus $g$, and a set $C_i$, $1 \leq i \leq n$, of
        $n$ pairwise disjoint simple closed curves embedded in the boundary of $H$, together with a
        complete set of cutting disks $D$ for $H$, such that when the members of $C$ and $D$ are
        appropriately labelled and oriented, $C_i$ freely and cyclically reduces to a cyclic
        conjugate of $R_i$, for $1 \leq i \leq n$.

\remark Note that, if $D$ is a Heegaard diagram with genus greater than one, and $D$ 
        realizes a presentation $P$, then there are an infinite number of diagrams obtainable
        from $D$ by performing Dehn twists along curves which bound disks in the underlying
        handlebody $H$, such that each of these diagrams also realizes $P$.

\remark We will give an example later which shows why it is necessary to allow free and
        cyclic reductions in the definition of realizability unless $P$ has minimal length.

\note   The program always assumes that the ``realization" of a presentation is accomplished
        by adding 2-handles to the handlebody $H$ along the set of curves $C$, and then any
        resulting 2-sphere boundary components are ``capped-off" with 3-balls to yield a
        3-manifold $M$.
        
\note   The program does not keep track of the orientations of Heegaard diagrams, and so
        does not keep track of the orientations of the corresponding 3-manifolds. In particular,
        homeomorphisms may be either orientation preserving or orientation reversing.   

\remark Many 3-manifolds have ``inequivalent" Heegaard splittings. The program does not
        currently make any explicit attempt to search for such inequivalent splittings. They may
        arise naturally, however, as the program produces diagrams from an initial presentation
        by eliminating primitive relators in different ways. For example, this is essentially what
        occurs when the program finds distinct 1-relator presentations of the exteriors of tunnel-
        number-one knots, such as the $(-2,3,7)$ pretzel knot.
        
\remark The program contains routines which will recognize the 3-sphere and lens spaces under
        certain conditions. These routines are not claimed to be algorithmic. On the other hand,
        we do not currently know of any examples for which these routines fail. It would, of
        course, be very interesting to have such examples.      
                
\remark The procedures which the program uses to generate new presentations are often not
        invertible, or at least not easily inverted. This is one of the major reasons that the
        program saves what eventually turn out to be superfluous presentations. These presentations
        are saved in case the program runs into a dead end and needs to backup to a previous
        stage in order to continue.

\definition     Let $P$ be a presentation, and let $R$ be a relator which appears in $P$. The ``algebraic
        length" of $R$ is the total number of letters which appear in $R$ after $R$ is freely and
        cyclically reduced. The ``algebraic length" $|P|$, of $P$ is the sum of the algebraic lengths
        of all of the relators which appear in $P$.

\definition     Let $D$ be a Heegaard diagram. The ``geometric length" $|D|$, of $D$ is the total number of
        edges which appear in $D$.      

\notation       We will often use upper-case and lower-case letters to denote generators and their
        inverses in presentations.

\definition     A ``defining relator" is a relator $R$, which is freely and cyclically reduced, and which
        has the property that there exists a generator say $G$, such that the total number of
        appearances of $G$ and $g$ in $R$ is one.
        
                Recall that this allows $R$ to be used to express $G$ in terms of the other generators
        appearing in $R$. Then the resulting expression can be used to eliminate all appearances of
        $G$ and $g$ from the remaining relators.

\definition     A ``primitive" is a relator which represents a free generator of the free group
        generated by the generators of the presentation.

                By ``deleting" or ``eliminating" free generators, the program can reduce the rank of
        a free group and reduce the Heegaard genus of the associated manifold.
                                
\definition     A Heegaard diagram $D$, of a closed orientable 3-manifold $M$, is ``pseudo-minimal" if
        there do not exist any automorphisms which reduce the length of either of the two dual
        presentations associated with $D$.

\definition     Suppose $G$ is a graph and $V$ is a vertex of $G$. The ``link" of $V$ in $G$ is the set of
        vertices of $G$ which are joined to $V$ by one or more edges.
                                                
\definition     Suppose $G$ is a graph. A ``component" of $G$ is a subset $C$, of the vertices of $G$, which
        is maximal with respect to the property that any two vertices in $C$ can be joined by a
        path in $G$.

\definition     Suppose $V$ is a vertex of a graph $G$, and $G$ has $n$ components. If deleting $V$ from $G$
        yields a graph $G'$ which has more than $n$ components, then we say $V$ is a ``cut vertex"
        of $G$.
        
\definition     Let $P$ be a presentation on generators $\{ A,B,C,  \ldots \}$, such that the relators of $P$
        are freely and cyclically reduced. (Assume $P$ does not contain any empty relators.)
        If $X$ is a generator, let $x$ denote the inverse of $X$. The ``Whitehead graph" of $P$ is the
        graph with vertices $\{ A,a,B,b,C,c,  \ldots \}$, and an undirected edge joining vertex $x$ to
        vertex $Y$ for each occurrence of the pair of letters $XY$ in the relators.
        
\note   It will be useful to think of the Whitehead graph of a presentation $P$ on $g$
        generators and $n$ freely and cyclically reduced relators, as the graph obtained in the
        following way:
        \begin{enumerate}
                \item      Let $H$ be a handlebody of genus $g$ with a complete set of cutting disks $D$.
                \item     Orient and label the members of $D$ so they correspond to the generators of $P$.
                \item      Embed $n$ pairwise disjoint simple closed curves in the interior of $H$ so that
                        they ``represent" the freely and cyclically reduced relators of $P$, taking care
                        that the total number of intersections of this set of curves with the members
                        of $D$, equals the total length of the relators of $P$.
                \item     Cut $H$ open along the members of $D$ to obtain a 3-ball $B$ with $2g$ disks in its
                        boundary.
                \item     Identify the $2g$ disks in the boundary of $B$ with the vertices of the Whitehead
                        graph.
                \item     When $H$ is cut open along the members of $D$, the simple closed curves that
                        represent the relators of $P$ are cut into a set of arcs. Identify these arcs
                        with the edges of the Whitehead graph.
        \end{enumerate}
        Note that in general, there are multiple edges joining any given pair of vertices in
        the Whitehead graph. It will often be convenient to pass to a simpler graph which does
        not have any pair of its vertices joined by more than one edge.
        
\definition     Let $WG$ be the Whitehead graph of a presentation $P$. The ``reduced Whitehead graph"
        of $P$, is the graph $RWG$ obtained by replacing any sets of multiple edges, joining a pair
        of vertices of $WG$, with a single edge joining that pair of vertices.

                                        
                                                
\section{THE BASIC THEORY WHICH UNDERLIES THE PROGRAM.}
                                                
                                                        
                Most of basic theory which underlies the program is fairly simple, and has its roots
        in fundamental results about automorphisms of finitely generated free groups due to J.H.C.
        Whitehead in 1936, together with more recent results of Zieschang, which show that the
        algebraic results of Whitehead can be realized topologically in the case of Heegaard
        diagrams.

                While the results of Whitehead and Zieschang form the mathematical basis for
        much of the program, the actual development of a program which effectively exploits
        their results requires more. In particular, it must be possible to find and perform
        certain simplifications of Heegaard diagrams, which Whitehead and Zieschang guarantee
        exist, in an efficient manner. This is important, because although efficiency is largely
        irrelevant from a mathematical point of view, it is a crucial consideration when it comes
        to the development of a practical program.
        
        
\section{WHITEHEAD'S RESULTS ABOUT AUTOMORPHISMS OF FINITELY GENERATED FREE GROUPS.}
                        
        
\definition     Let $F$ be a finitely generated free group, and let $G$ be a set of free generators for
        $F$. An automorphism $\phi$ of $F$ is a ``T-transformation" if $\phi$ permutes the generators in $G$, or
        if, for some fixed generator $A$ in $G$, $\phi$ carries $A$ to $A$, and $\phi$ carries each of the
        remaining generators $X$ in $G$ into one of $X, AX, Xa$ or $AXa$.
        
\note   There are only a finite number of T-transformations in the automorphism group of $F$.
        Indeed it is not hard to see that the number of nontrivial T-transformations which are
        not permutations of the generators, in a free group of rank $g$, $g > 1$, is $g(4^{g-1}-2)$
        up to conjugacy. Note further, however, that although this number is finite, it grows
        exponentially in the rank of $F$.       
        
\theorem(Whitehead) Let $P$ be a finite presentation on a finite set $G$ of $g$ generators such
        that the relators of $P$ are freely and cyclically reduced. Let $F$ be the free group freely
        generated by the members of $G$. If there exists an automorphism $\pi$  of $F$ which reduces the
        length of $P$, then there exists a T-transformation $\phi$ of $F$ which also reduces the length
        of $P$.

\definition     A presentation $P$ has ``minimal length" if the relators of $P$ are freely and cyclically
        reduced, and there do not exist any automorphisms of the free group $F$, freely generated
        by the generators of $P$, which reduce the length of $P$.

\definition     Suppose $P$ is a presentation which has minimal length. If $\phi$ is a T-transformation
        which transforms $P$ into a presentation $P'$, such that $P$ and $P'$ have the same length,
        then $\phi$ is a ``level-T-transformation" or simply   ``level-transformation".
                
\theorem (Whitehead) Let $P$ and $P'$ be finite presentations on a finite set $G$ of $g$ generators
        such that the relators of both $P$ and $P'$ are freely and cyclically reduced. Let $F$ be the
        free group freely generated by the members of $G$. Suppose both $P$ and $P'$ have minimal
        length, and suppose there exists an automorphism $\pi$  of $F$ such that $\pi(P) = P'$. Then there
        exists a finite sequence of level-T-transformations which also transforms $P$ into $P'$.
        
\corollary      (Whitehead) Let $P$ be a finite presentation on a finite set $G$ of $g$ generators such
        that the relators of $P$ are freely and cyclically reduced. Let $F$ be the free group freely
        generated by the members of $G$. There is an effective procedure which finds a sequence of
        automorphisms of $F$ that reduce $P$ to minimal length.
        
\corollary      (Whitehead) Let $P$ be a finite presentation on a finite set $G$ of $g$ generators such
        that the relators of $P$ are freely and cyclically reduced. Suppose $P$ has minimal length.
        Let $F$ be the free group freely generated by the members of $G$. Then there exists an
        effective procedure to generate and enumerate the set of all minimal length
        presentations into which $P$ can be transformed by automorphisms of $F$.        
        

\section{SOME RESULTS OF ZIESCHANG WHICH RELATE WHITEHEAD'S RESULTS TO HEEGAARD DIAGRAMS.}

        
\definition     Let $D$ be a Heegaard diagram, and let $H$ be the underlying handlebody. Suppose $H$ has
        genus $n$ and that $d_i, 1 \leq i \leq n$, is a complete set of cutting disks for $H$. We assume
        that the intersections of the curves of the diagram with the cutting disks are minimal
        up to isotopy in the boundary of $H$. Suppose $d$ is a nonseparating disk in $H$ such that
        $d$ is disjoint from $d_i$ for $1 \leq i \leq n$. If it is possible to replace one of the $d_i$(s)
        with $d$ so as to obtain a complete set of cutting disks for $H$ which has fewer
        intersections with the attaching curves of the 2-handles of $D$, then we say that a
         ``geometric-T-transformation" has occurred.

\theorem (Zieschang) Suppose $D$ is a Heegaard diagram with underlying handlebody $H$ and P
        is the presentation of the fundamental group associated with the handlebody $H$. If there
        exists a T-transformation which reduces $|P|$, then there exists a geometric-T-
        transformation which reduces $|D|$.
        
\corollary      (Zieschang) Suppose $D$ is a Heegaard diagram with underlying handlebody $H$ and P
        is the presentation of the fundamental group associated with the handlebody $H$. Suppose
        that $P'$ is any minimal length presentation obtained from $P$ via a finite sequence of
        T-transformations, then there exists a finite sequence of geometric T-transformations
        which transform $D$ into a Heegaard diagram $D'$ such that $|D'| = |P'|$ and $D'$ realizes $P'$.

\definition     Suppose $D$ is a Heegaard diagram with underlying handlebody $H$ and $P$ is the
        presentation of the fundamental group associated with the handlebody $H$. Suppose $P$ has
        minimal length, $|D| = |P|$, and $\Sigma$ is a geometric-T-transformation which carries
        $D$ into $D'$ such that $|D'| = |D|$. Then we say $\Sigma$ is a  ``level-geometric-T-transformation".     
        
\corollary      (Zieschang)      Suppose $D$ is a Heegaard diagram with underlying handlebody $H$ and $P$
        is the presentation of the fundamental group associated with the handlebody $H$. Suppose
        $P$ has minimal length, $|D| = |P|$, and $P'$ is a presentation obtained from $P$ by a
        level-T-transformation. Then there exists a level-geometric-T-transformation which
        carries $D$ into $D'$ such that $|D'| = |P'|$, and $D'$ realizes $P'$.
        
\remark Suppose $D$ is a Heegaard diagram with underlying handlebody $H$ and $P$ is the
        presentation of the fundamental group associated with the handlebody $H$. Suppose that
        $\phi$ is a T-transformation which reduces the length of $P$. Zieschang's results guarantee
        that there is a geometric-T-transformation $\Sigma$ which reduces the geometric length of $D$.
        Now $\Sigma$ induces some T-transformation of $P$, but it is not necessarily the case that
        $\Sigma$ = $\phi$. Indeed, there may be no geometric-T-transformation which induces $\phi$. It is also
        not hard to produce examples where any geometric-T-transformation which reduces $|D|$,
        must actually increase $|P|$.
                

\section{REFERENCES}

\begin{enumerate}
\item      J.H.C. Whitehead, On equivalent sets of elements in a free group, Ann. of Math.
        (2)37(1936), 782-800.                                                   
\item     Zieschang, On simple systems of paths on complete pretzels. Amer.Math.Soc.Transl.
        (2) Vol.92,1970, 127-137
\end{enumerate}

              
 \section{HOW THE PROGRAM INSURES THAT THE DIAGRAMS IT PRODUCES ARE ALL HEEGAARD DIAGRAMS OF THE SAME 3-MANIFOLD.}

                
                It is obviously important that the Heegaard diagrams which the program produces
        all be Heegaard diagrams of the same 3-manifold. In order to guarantee that this is
        the case, new presentations must be obtained from previous presentations only in
        certain ways. Here we describe how the program does this, and we indicate when it is
        possible to recover a Heegaard diagram of the original 3-manifold from the
        transformed presentation.
        
                Suppose $P$ is a realizable presentation of a 3-manifold $M$. The program is
        allowed to produce a new presentation $P'$ from $P$ in one of following four ways,
        provided the stated conditions hold.            
        \begin{enumerate}
                \item     If $P$ does not have minimal length, let $P'$ be any presentation obtained
                        from $P$ by reducing $P$ to minimal length.
                        
                \item      If $P$ has minimal length, let $P'$ be any presentation obtained from $P$ by
                        a sequence of level-transformations.
                        
                \item      If $P$ is a minimal length balanced presentation, $P$ has a unique realization
                        as a Heegaard diagram $D$, where $|D| = |P|$, and $D$ is a Heegaard diagram of a
                        closed manifold, let $\hat P$ be the dual presentation corresponding to $D$. Then
                        let $P'$ be any presentation obtained from $\hat P$ by reducing $\hat P$ to minimal length.
                        
                \item      If $P$ is a minimal length presentation, $P$ has a unique realization as a
                        Heegaard diagram $D$, where $|D| = |P|$, and $R$ is a relator of $P$, let $\hat P$ be a
                        presentation obtained from $P$ by replacing $R$ with a new relator $R'$, where $R'$
                        is a geometrically realizable bandsum of $R$ and another relator $R"$ of $P$. Then
                        let $P'$ be any presentation obtained from $\hat P$ by reducing $\hat P$ to minimal length.             
         \end{enumerate}
        The next lemma follows from the preceding results of Whitehead and Zieschang.

\lemmaone       Suppose $P$ is a realizable presentation of a 3-manifold $M$. Suppose that $P''$
        is a presentation obtained from $P$ by reducing $P$ to minimal length or $P$ has minimal
        length and $P''$ is obtained from $P$ via a sequence of level-transformations. Then $M$ has
        a Heegaard diagram $D''$ such that $D''$ realizes $P''$ and $|D''| = |P''|$.
                
\lemmatwo        Suppose $P$ is a realizable presentation of a 3-manifold $M$. Suppose that $P'$
        is a presentation obtained from $P$ by reducing $P$ to minimal length or $P$ has minimal
        length and $P'$ is obtained from $P$ via a sequence of level-transformations. Suppose
        furthermore, that there is a unique Heegaard diagram $D'$, with $|D'| = |P'|$, which
        realizes $P'$. Then $D'$ is a Heegaard diagram of $M$.
        
\pf  This is an immediate consequence of the existence provided by Lemma 1) and the
        uniqueness provided by Lemma 2).        

        Applying these lemmas to the four situations above yields the following theorem.
        
\theorem        Suppose $P$ is a realizable presentation of a 3-manifold $M$. Suppose that $P'$ is
        a minimal length presentation obtained from $P$ in one of the four ways listed above,
        and suppose there is a unique Heegaard diagram $D'$, with $|D'| = |P'|$, which realizes $P'$.
        Then $D'$ is a Heegaard diagram of $M$.

\remark The program's manipulations of presentations are based largely on this Theorem.
        In particular, this means that the program places a premium on obtaining minimal length
        presentations whose realizations by Heegaard diagrams, with geometric length equal to
        the algebraic length of the presentation, are unique.
        
\remark Suppose $P$ is a minimal length presentation and there are several distinct Heegaard
        diagrams, with geometric length equal to $|P|$, which realize $P$. Then one of the
        diagrams which realizes $P$ is a diagram of $M$. However, determining which realization
        is  ``correct" requires additional information. At this point, we have not settled on an
        efficient scheme which the program can use to do this.
        
                
\section{AN EFFICIENT METHOD FOR FINDING T-TRANSFORMATIONS THAT REDUCE THE LENGTH OF A  PRESENTATION.}
        
As has been alluded to, a practical program needs an efficient method
for locating and performing T-transformations which reduce the length
of a presentation. We describe such a procedure below. The fact that
such an efficient procedure exists does not seem to be well-known. We
also wish to point out that the procedure we will describe can be
applied to any presentation, and so it may well have other
applications elsewhere.  For these reasons, we will give a fairly
detailed and complete description of the procedure.
                
                The key to finding T-transformations which will reduce the length of a given
        presentation $P$ efficiently, is to look at the problem of finding such a T-
        transformation as one of finding a minimal cut set of edges in the Whitehead graph of
        $P$, where the cut set of edges separates the vertices corresponding to the  ``operating"
        generator and its inverse from each other. This reduces the problem of finding a
        T-transformation that will reduce the length of $P$ to a well-understood problem
        in graph theory of finding a maximal flow in a network. There are extremely efficient
        methods known for solving this type of problem.
        
                Here are some of the details. Suppose $P$ is a finite presentation on a finite set $G$
        of $g$ generators such that the relators of $P$ are freely and cyclically reduced. Let $F$ be
        the free group freely generated by the members of $G$. Let $WG$ be the Whitehead graph of $P$,
        and let $RWG$ be the reduced Whitehead graph of $P$. Let $A$ be a generator in $G$, and let a be
        its inverse. Suppose that $\phi$ is a T-transformation which is not a permutation of the
        generators in $G$. Then if is $X$ is a generator in $G$, $\phi$ acts on $X$, (up to conjugation by $a$),
        in one of the following four ways: $X \to   AX$, $X \to  Xa$, $X \to AXa$ or $X \to X$.
        
        \lemma  There is a one-to-one correspondence between T-transformations in which the
        generator $A$ acts as a multiplier on the remaining generators of $G$, (up to conjugacy)
        and partitions of the vertices of the Whitehead graph of $G$ into two disjoint subsets,
        $SA,Sa,$ where $A$ lies in $SA$, and $a$ lies in $Sa$.
        
        \pf  Define a correspondence of the required type by the following rule:
        \begin{enumerate}
                \item      A lies in $SA$, and a lies in $Sa$.
                 \item $\phi(X) = AX$  if and only if $X$ lies in $SA$, and $x$ lies in $Sa$.
                \item      $\phi(X) = Xa$  if and only if $X$ lies in $Sa$, and $x$ lies in $SA$.
                \item      $\phi(X) = AXa$ if and only if both $X$ and $x$ lie in $SA$.
                \item      $\phi(X) = X$   if and only if both $X$ and $x$ lie in $Sa$.
              \end{enumerate}

                Now, consider the Whitehead graph $WG$ of $P$ as being embedded in a 3-ball $B$ with $2g$
        disjoint disks on $\partial B$, which can be identified in pairs to yield a handlebody $H$ of genus
        $g$, as described above. Let $SA$ and $Sa$ be the partition of the vertices of $WG$ determined
        by $\phi$ using the rule given by the preceding lemma. Observe that we may consider $B$ as
        being divided into two hemispheres $N$ and $S$, by an equatorial disk $D$, such that all of
        the vertices of $SA$ lie in $N$, while all of the vertices of $Sa$ lie in $S$. We also suppose
        that the arcs which represent the edges of $WG$ are embedded in $B$ so that they intersect $D$
        minimally, up to homotopy in $B$, keeping their endpoints fixed.

                Choose a point $b \in S$, to serve as the basepoint of the fundamental group of $H$. Then,
        for each pair of vertices $\{ $X$,x\}$ of $WG$, choose an oriented path which leaves $b$, goes to
        the disk on $\partial B$ representing $X$, and returns to $b$ from the disk on $\partial B$ which represents $x$.
        Suppose these paths are chosen so that they intersect $D$ minimally. Clearly, these paths
        form a set of generators for $\pi_1(H)$, which we may view as representing the current set of
        generators.
        
        Next, observe that since the disk $A$ lies in $N$, and the
        disk a lies in $S$, we can replace the cutting disk
        corresponding to the generator$A$, with the disk $D$. Suppose
        this is done. The result is a new set of cutting disks for
        $H$, with the change of cutting disks inducing an automorphism
        $\mu$ of $\pi_1(H)$, and we see that if $D$ is relabelled as
        $A$, then $ \mu = \phi$.

                Observe that the number of times generator $A$ and its inverse a appear in $P$, in this
        new basis for $\pi_1(H)$, is completely determined by the number of edges of $WG$ which join
        vertices in $SA$ to vertices in $Sa$, while the number of appearances of the remaining
        generators in $P$ is unchanged. Observe also, that because $WG$ corresponds to $P$, and $P$ is
        freely and cyclically reduced, all intersections of the simple closed curves in $H$, which
        represent relators of $P$, are essential intersections with this new set of cutting disks
        for $H$.
        
                The following definitions will now be useful.
        
\definition     If $X$ is a vertex of a graph $G$, let $V(X)$ denote the number of edges of $G$ meeting $X$.
        
\definition Let $G$ be a graph, $X$ and $Y$ vertices of $G$, and let
$C$ be a subset of the edges of $G$.  Let $G'$ be the graph obtained
by deleting the edges in $C$ from $G$. We say that ``C is a cut-set of
edges of $G$ separating $X$ and Y", if $X$ and $Y$ lie in different
connected components of $G'$.

\definition     Let $C$ be a cut-set of edges of a graph $G$. Define the  ``capacity of C" to be the number
        of edges which $C$ contains.
                
\definition     Let $G$ be a graph, $X$ and $Y$ vertices of $G$, and let $C$ be a cut-set of edges of G
        separating $X$ and $Y$. We say $C$ is a  ``minimal cut-set of edges separating $X$ and Y", if
        no subset $D$, of the edges of $G$, such that $D$ contains fewer edges of $G$ than $C$ contains,
        separates $X$ and $Y$.

                Note that a cut-set of edges separating $X$ and $Y$ may be empty.
        
                We can now prove the main theorem of this section.
        
\theorem  Suppose $P$ is a finite presentation on a finite set $G$ of $g$ generators such that the
        relators of $P$ are freely and cyclically reduced. Let $F$ be the free group freely generated
        by the members of $G$. Let $WG$ be the Whitehead graph of $P$. Let $A$ be a generator in $G$, and
        let $a$ be the inverse of $A$. There exists a T-transformation which reduces the number of
        appearances of $A$ and its inverse in $P$, if and only if there exists a cut-set $C$ of edges
        of the Whitehead graph $WG$, such that $C$ separates $A$ and its inverse $a$, and the capacity
        of $C$ is less than $V(A)$.

\pf  The preceding discussion shows that if there is a T-transformation which reduces
        the number of appearances of the generator A together with its inverse in $P$, then there
        is a cut-set of edges $C$ of $WG$ such that $C$ separates vertex $A$ from vertex $a$, and the
        capacity of $C$ is less than $V(A)$.
             
        Conversely, suppose that $C$ is a cut-set of edges of $WG$ which separates vertex $A$ from
        vertex a in $WG$, and suppose the capacity of $C$ is less than $V(A)$. Let $G'$ be the graph
        obtained when the edges of $C$ are deleted from $WG$. Let $SA$ be the set of vertices of the
        component of $G'$ containing vertex $A$, and let $Sa$ be the remaining vertices of $G'$. Define
        a T-transformation $\phi$, in which $A$ acts on the remaining generators in $G$, by using the rule
        given in the preceding lemma above. Now consider $WG$ to be embedded in a 3-ball $B$, as
        in the preceding discussion. We see that $\phi$ reduces the number of appearances of $A$ and
        its inverse, in $P$.
                
                For the next theorem, we will assume that the Whitehead graph of $P$ is connected.
        We do this only to simplify the statement of the theorem. The appropriate statement, in
        the case where the Whitehead graph of $P$ is not connected, is not hard to find.
                
\theorem  Suppose $P$ is a finite presentation on a finite set $G$ of $g$ generators such that the
        relators of $P$ are freely and cyclically reduced. Let $F$ be the free group freely generated
        by the members of $G$. Suppose that $P$ has minimal length. And suppose the Whitehead graph
        WG of $P$ is connected. Let $A$ be a generator in $G$, and let $a$ be the inverse of $A$. There
        exists a non-trivial level-T-transformation in which A acts on the remaining generators
        of $G$, if and only if there is a cut-set $C$ of edges of the Whitehead graph $WG$ such that $C$
        separates $A$ and its inverse $a$, the capacity of $C$ equals $V(A)$, and neither component, of
        the graph obtained by deleting the edges of $C$ from $WG$, consists of only a single vertex.
                
\pf  Observe that the graph $G'$ obtained by deleting the edges of $C$ from $WG$, has only two
        components. Let $SA$ be the set of vertices of the component of $G'$ containing vertex A, and
        let $Sa$ be the set of remaining vertices of $G'$. Use $SA$ and $Sa$ to define a T-transformation,
        and proceed as before.          
        
        Up to this point, we have only shown that there is a strong
        connection between T-transformations, and cut-sets of edges.
        In particular, we have seen that if we wish to reduce the
        number of appearances of say, generator $A$ and its inverse in a
        presentation $P$, then we should look for a cut-set $C$, of
        edges of the Whitehead graph $WG$ of $P$, such that $C$
        separates vertex $A$ from vertex $a$ in $WG$, and $C$ has capacity
        less than $V(A)$. But, the number of appearances of generator $A$
        and its inverse in the transformed presentation is equal to
        the capacity of $C$. This suggests we should be ``greedy" and
        look for minimal cut-sets of edges separating vertex $A$ from
        vertex $a$. This turns out to be the right thing to do, because
        there are very efficient ways to find such minimal cut-sets.
        
        \section{FINDING T-TRANSFORMATIONS EFFICIENTLY BY FINDING MAXIMAL FLOWS IN NETWORKS.}

                The efficient way to find minimal cut-sets and hence T-transformations, is to
        dualize the problem and to look for a  ``maximal flow" in the network given by the
        Whitehead graph of $P$. This problem has been intensively studied in many areas,
        including computer science, graph theory, and operations research, so it should not
        be hard to find numerous references. Since this is the case, we will only include a
        sketch of the basic ideas.
                
                Suppose we wish to find a minimal cut-set of edges separating vertex $A$ from vertex
        a in the Whitehead graph of some presentation $P$. Consider the vertex $A$ to be the
         ``source" vertex and vertex a to be the  ``sink" vertex of a network given by $WG$. Now the
        idea is to find a maximal collection of directed paths in $WG$, such that each path runs
        from the source to the sink, and such that no two paths have any edges in common and no
        path traverses a given edge more than once. Let $F$ be such a maximal collection. Then the
        duality Theorem of Ford and Fulkerson asserts that the number of paths in $F$ equals the
        capacity of a minimal cut-set of edges separating source and sink.
          
        Suppose that $F$ is given, then there is no directed path, which joins source to sink,
        and which does not have directed edges, with identical directions, in common with paths
        in $F$. Let $SA$ be the set of vertices of $WG$, which are accessible from the source vertex $A$
        by directed paths which share no directed edges, with identical directions, with members
        of $F$. Let $Sa$ be the set of remaining vertices of $WG$. Then the edges which join vertices
        in $SA$ to $Sa$ form a minimal cut-set of edges of $WG$ separating vertex $A$ from vertex $a$.

                As far as efficiency goes, we mention only that there are ways to find  ``maximal flows"
        and, by duality, minimal cut-sets in a graph with $V$ vertices and $E$ edges in time which is
        at worst $O(VE(\log V))$.
        
\remark Note that these same ideas can be applied, essentially without change, to find
        T\hyp transformations of sets of  ``true" words in a free group. In this case, the Whitehead
        graph has an additional vertex $b$, which serves as a basepoint for all of the paths
        involved, but aside from this, there is no basic difference.
        

                                                        \section{A PROOF OF ZIESCHANG'S THEOREM}
                                                        
                At this point, we can give a proof of Zieschang's Theorem which uses the idea of
        looking for minimal cut-sets of the Whitehead graph of a presentation. The proof will
        show that, under most circumstances, the T-transformations, which the program uses to
        simplify presentations, are actually geometric.
        
        \theorem (Zieschang) Suppose $D$ is a Heegaard diagram with underlying handlebody $H$ and P
        is the presentation of the fundamental group associated with the handlebody $H$. If there
        exists a T-transformation which reduces the length of $P$, then there exists a geometric-
        T-transformation which reduces the geometric length of $D$.
        
        \pf We assume $P$ is freely and cyclically reduced. There are two cases: 1) $|D| > |P|$,
        2) $|D| = |P|$.
                Suppose $|D| > |P|$. Then there exists an edge $E$ of $D$ and a vertex $X$ of $D$ such that
        both ends of $E$ lie on $\partial X$. Let $N$ be a regular neighborhood in $D$ of $X$ union $E$. Then $\partial N$
        consists of two simple closed curves, one of which, say $C$, separates $X$ from its inverse
        x, in $D$. Then replacing the cutting disk $X$ with a disk in $H$, whose boundary is $C$, is a
        geometric-T-transformation which reduces $|D|$.
                So we may suppose that $|D| = |P|$. Then no edge of $D$ has both of its endpoints on
        the same vertex of $D$. Thus $D$ is essentially an embedding of the Whitehead graph $WG$ of $P$,
        in the plane.

                Now suppose that $\phi$ is a T-transformation in which generator A acts on the remaining
        generators of $P$, $|\phi(P)| \leq |P|$, and that no other T-transformation, in which $A$ acts on
        the remaining generators of $P$, reduces $|P|$ more than $\phi$ does. Let $SA,Sa$ be the partition
        of the vertices of $WG$ induced by $\phi$ using the correspondence described in the lemma above.
        Let $C$ be the cut-set of edges of $WG$ consisting of those edges of $WG$ which join vertices in
        $SA$ to vertices in $Sa$. Then $C$ is a minimal cut-set of edges of $WG$ separating vertex $A$ and
        vertex $a$.

                Now consider cutting the set of edges of $D$ which correspond to the edges in $C$. Note
        that the minimality of $C$ implies that the number of components of $D$ increases by at most
        one. Let $CA$ be the component of $D - C$, which contains vertex $A$. Let $N$ be a regular
        neighborhood in $D$ of $CA$. Then $N$ is a connected planar surface, and the closure of each
        complementary domain of $N$ is a disk. Let $Da$ be the disk which contains vertex $a$ in its
        interior. Observe that minimality of $C$ implies that the only edges of $WG$ which intersect
        $\partial N \cap \partial Da$. Replace $\partial A$ with $\partial Da$. Note this is a geometric-T-transformation.
        
                Then we have the following corollary.
        
        \corollary Suppose $D$ is a Heegaard diagram with underlying handlebody $H$, $P$ is the freely and
        cyclically reduced presentation of the fundamental group associated with the handlebody
        $H$, and $|D| = |P|$. Suppose that $\phi$ is a T-transformation in which generator $A$ acts on the
        remaining generators of $P$, $|\phi(P)| \leq |P|$, and that no other T-transformation, in which $A$
        acts on the remaining generators of $P$, reduces $|P|$ more than $\phi$ does. If $D$ is connected,
        there exists a geometric-T-transformation $\Sigma$ which induces $\phi$. If $D$ is not connected, there
        exists a geometric-T-transformation $\Sigma$, such that $\Sigma$(P) = $\phi$(P).
        
        \pf   If $D$ is connected, all of the vertices of $WG$ except those in $CA$, lie in the
        interior of $Da$. Thus $\Sigma$ induces the same partition of the vertices of $WG$ as does $\phi$, and
        so $\Sigma$ induces $\phi$. If $D$ is not connected, there may be vertices of $WG$ which lie in other
        components of the complement of $N$. However, the action of a T-transformation on $P$ is
        completely determined by the set $C$ of edges which join the vertices in the set $SA$ to 
        vertices in the set $Sa$. This set is identical in both cases. Hence, even though we need
        not have $\Sigma$ = $\phi$, it is still the case that $\Sigma$(P) = $\phi$(P).
        
                So we see that, in most cases of interest, e.g.~when $|D| = |P|$ and the Whitehead graph
        of $P$ is connected, the T-transformations, which the program uses to modify presentations,
        are geometric.



\section{ATTEMPTING TO RECOVER A HEEGAARD DIAGRAM FROM A MINIMAL LENGTH PRESENTATION.}
                
                Suppose $P$ is a minimal length presentation which is realizable and the program
        wishes to produce a Heegaard diagram $D$ which realizes $P$. Producing $D$ involves several
        steps. The first step is motivated by the following result which follows from the
        results of Whitehead and Zieschang.
        
\lemma  Suppose $P$ is a minimal length presentation which is realizable and $WG$ is the
        Whitehead graph of $P$. Then $WG$ is planar.
        
\pf   By the results of Whitehead and Zieschang above, there exists a Heegaard diagram
        D which realizes $P$ such that $|D| = |P|$. If we consider $D$ as a graph by identifying 
        the disks of $D$ to points, then the resulting graph is an embedding of the Whitehead
        graph $WG$ of $P$ in the plane.

\remark This lemma is false without the stipulation that $P$ has minimal length. That is,
        there exist realizable presentations which do not have minimal length, and whose
        Whitehead graphs are not planar. We will give an example later. 
                
                Now the program is essentially in the position of attempting to recover $D$ from
        P, so the program's first step is to obtain the Whitehead graph $WG$ of $P$, and then
        to pass to the reduced Whitehead graph $RWG$ of $P$. It follows from the preceding lemma
        that $WG$ and hence $RWG$ must by planar. The next step is to find an embedding of $RWG$ in
        the plane. Suppose for the moment that $RWG$ has a unique embedding in the plane. Then
        the embedding of $RWG$ can be extended to an embedding of $WG$ in the plane. This extension
        of the embedding of $RWG$ to an embedding of $WG$ is unique. (See the section titled
        THE DEFINITION OF  ``PAIRS OF SEPARATING VERTICES" for an explanation as to why this is
        the case.) If we then replace the vertices of $WG$ with small disks, we have recovered $D$
        except for information about how pairs of disks corresponding to pairs of inverse
        vertices of $WG$ should be identified. The last step is to determine how these pairs of
        inverse disks should be identified. The basic procedure, which the program uses to do
        this, is described a following section.

                We now briefly consider the situation where $RWG$ does not have a unique embedding
        in the plane. Suppose $P'$ is a presentation obtained from $P$ via a sequence of level-
        transformations. Then $P'$ is realizable, and there exists a Heegaard diagram $D'$ which
        realizes $P'$ with $|D'| = |P'|$. If the $RWG$ of $P'$ embeds uniquely in the plane, the
        program will try and recover the diagram $D'$ instead of $D$. Generally, $D'$ serves as well
        as $D$, so not having $D$ itself is no great loss. It is a fortunate situation from the
        program's point of view, that if the reduced Whitehead graph of a presentation does not
        embed uniquely in the plane, then there exist level-transformations. It is also a
        fortunate situation from the program's point of view, that when this situation occurs,
        there is generally a sequence of level-transformations which leads to a presentation
        whose reduced Whitehead graph embeds uniquely in the plane. Since level-transformations
        seem to be fairly effective in this situation, the program has made the use of level-
        transformations a major part of the tactics which it uses in order to find a diagram
        that realizes a given presentation.

                However, the exact set of circumstances under which it is possible to pass via
        level-transformations from a presentation whose reduced Whitehead graph does not embed
        uniquely in the plane, to a presentation whose reduced Whitehead graph does embed
        uniquely, are not completely clear. This question is discussed in more detail below.    

        
\section{ EXAMPLE: A PRESENTATION $P$, WITH THE PROPERTY THAT ANY REALIZATION OF $P$ HAS GEOMETRIC LENGTH GREATER THAN $|P|$.}

                Here is the example promised above. We like this example because it helps to
        justify the effort the program makes to reduce presentations to minimal length,
        before it attempts to determine whether they are realizable.
        
        Let $P$ be the following presentation whose relators are freely and cyclically reduced.
        \begin{enumerate}
                \item[R1:]           AAAAABBAACCC
                \item[R2:]           AAABBBCCCCC
                \item[R3:]           AABDCCBD
         \end{enumerate}
        If you obtain the reduced Whitehead graph $RWG$, of $P$, you will see that $RWG$ is not
        planar. This shows that if there exists a Heegaard diagram $D$, which realizes $P$, then
        $|D| > |P|$.

                If $P$ had minimal length, the fact that the reduced Whitehead graph of $P$ is not
        planar would imply that $P$ is not realizable. However, $P$ does not have minimal length,
        since the automorphism $D \to bD$ reduces the length of $P$ by 2. Applying this automorphism
        yields the following presentation $P'$:
       \begin{enumerate}
                \item[R1:]           AAAAABBAACCC
                \item[R2:]           AAABBBCCCCC
                \item[R3:]           AADCCD
        \end{enumerate}                
        Now you can easily verify that $P'$ has minimal length and $P'$ is realizable, which,
        of course, implies that $P$ is realizable.


        
        \section{THE STRATEGY WHICH THE PROGRAM USES, WHEN IT ATTEMPTS TO DETERMINE WHETHER A PRESENTATION IS REALIZABLE.}

                Before discussing the procedure which the program follows when it attempts to
        determine whether a presentation is realizable, it is important to point out that 
        determining whether a presentation is realizable and finding all possible realizations
        if they exist, is essentially solvable. This is another of Zieschang results. Some
        care is necessary here because of the fact that one can obtain many different diagrams
        that realize a presentation by performing Dehn twists about curves that bound disks in
        the underlying handlebody. To avoid this problem, we observe that if $P$ is a presentation
        which we wish to check for realizability, and $P'$ is any presentation obtained from $P$ by
        reducing $P$ to minimal length, then $P$ is realizable if and only if $P'$ is realizable.               
        
\theorem (ZIESCHANG) Suppose $P$ is a finite presentation which is freely and cyclically
        reduced. Suppose that $P'$ is obtained from $P$ by reducing $P$ to minimal length. Then there
        is a constructive procedure to determine whether $P'$ is realizable, and to find all
        possible Heegaard diagrams $D$ such that $D$ realizes $P'$ and $|D| = |P'|$.
        
\pf   Let $WG$ be the Whitehead graph of $P'$. There are only a finite number of distinct
        embeddings of $WG$ in the plane up to homeomorphism. Then for each such embedding, and
        each pair of vertices of $WG$ which are inverses say $X$,$x$, there are only a finite number
        of distinct ways to identify the ends of the edges of $WG$ which meet $X$ with the ends of
        the edges of $WG$ which meet $x$.
        
                While this gives a constructive procedure for solving the realization problem, and
        a constructive procedure for finding all possible realizations of a minimal length
        presentation, its brute force approach is much too inefficient to be used in any but
        the simplest situations. Hence we seek more efficient procedures.
                We have not completely solved this problem. The procedures that the program
        currently uses seem to be quite effective in practice, but we do not know to what extent
        they provide an algorithm for determining realizability. We will discuss this more below
        after giving an outline of the procedures the program currently uses.   

                When the program tries to determine whether the original presentation passed to it is
        realizable, it first attempts to do so without reducing the number of generators involved.
        Then, if the program is not initially successful in determining whether the original
        presentation is realizable, the program will start looking for primitive relators which
        it can use to reduce the genus of the associated Heegaard diagram. However, some care is
        required here for the following reasons.

                Note that if $P$ is a realizable presentation, and $P'$ is any presentation obtained from P
        by deleting a sequence of primitive relators, then $P'$ is also realizable and $P'$ determines
        the same 3-manifold as $P$ does. So, if $P'$ turns out to be not realizable, then neither is $P$.
        On the other hand, if $P'$ is realizable, then one cannot conclude, in general, that $P$ is
        realizable. Under certain circumstances, however, one can infer the realizability of $P$ from
        the realizability of $P'$. This is the case, when $P'$ was obtained from $P$ by deleting what the
        program calls  ``trivial generators". Here is the definition of a  ``trivial generator".
                
        \definition     Let $P$ be a presentation, a generator $G$ appearing in $P$, is a  ``trivial generator" if:
        \begin{enumerate}
                        \item     The total number of appearances of $G$, together with its inverse $g$, in the
                                relators of $P$ is less than or equal to two.
                        \item      There is exactly one relator in $P$ of length one which has the form $G$ or $g$.        
        \end{enumerate}
        It is not hard to see that $P$ is realizable if and only if $P'$ is realizable, provided $P'$ is
        obtained from $P$ by deleting a sequence of  ``trivial generators".       So when the program looks
        for primitive relators, which it can use to reduce the number of generators appearing in the
        original presentation, it looks first for  ``trivial generators", and if it obtains a
        realizable presentation $P'$ from $P$ by deleting a sequence of  ``trivial generators", then the
        program asserts that the original presentation is realizable.

                If the program is not successful in determining whether $P$ is realizable by deleting a
        sequence of  ``trivial generators", then the program looks for other primitive relators which
        it can use to eliminate generators. In this case, if the program finds a presentation P'
        which is realizable, the program will assert that $P'$ is realizable, but the program will not
        assert that the original presentation $P$ is realizable.                
        
        Finally, here is an outline of the procedure the program uses to determine realizability.

\noindent        
\verb Check_Initial_Presentation_For_Realizability() \{ 
\begin{enumerate}
\item[1.]  Reduce $P$ to minimal length.
\item[2.]  Obtain the Whitehead graph $WG$, of $P$.
\item[3.]  Obtain the reduced Whitehead graph $RWG$, of $P$.
\item[4.]  Check whether $RWG$ is connected.
  \begin{enumerate}
  \item[A.]  If $RWG$ is not connected, the associated 3-manifold is a connected-sum. Split
    the relators of $P$ into two disjoint sets which correspond to presentations of
    the summands, and return.
  \item[B.] If $RWG$ is connected, continue.    
  \end{enumerate}
\item[4.]  Check $RWG$ for pairs of separating vertices.

  \begin{enumerate}
  \item[A.] If $RWG$ has pairs of separating vertices, call the
    program's routine  \\ 
     \verb Level_Transformations(), and have it
    look for new presentations by sliding the components of
    separations around.
    \begin{enumerate}
    \item[a.]  If \verb Level_Transformations() finds an annulus, report the existence of the
      annulus and return.
    \item[b.]  If \verb Level_Transformations() finds a presentation which has no pairs of
      separating vertices in its Whitehead graph, go to 5).
    \item[c.]  If \verb Level_Transformations() finds a presentation whose reduced Whitehead
      graph is nonplanar, report that the initial presentation is not realizable,
      and return. 
    \item[d.]  Otherwise, see if the current presentation has any  ``trivial generators" which
      can be eliminated.
          \begin{enumerate}
            
          \item[1.] If such  ``trivial generators" exist, eliminate some, and go to 1).
          \item[2.]  Otherwise, look for other relators which represent  ``primitives".
            
            A.  If such  ``primitives" exist, use some of them to reduce the rank
            of the presentation, and go to 1).
            
            B.  Otherwise, declare that the program cannot determine whether the
            presentation is realizable, and return.
          \end{enumerate}
        \end{enumerate}
      \item[B.]  If $RWG$ has no pairs of separating vertices, continue. 
      \end{enumerate}



    \item[5.]  Check $RWG$ for planarity.
      \begin{enumerate}
      \item[A.] If $RWG$ is not planar, declare $P$ not realizable and quit.
      \item[B.]  Otherwise, $RWG$ has a unique embedding in the plane. Find this embedding, extend
        the embedding to an embedding of $WG$, and continue.
        
        Note: Because of way the program defines pairs of separating vertices, the
        embedding of $WG$ is completely determined by the embedding of $RG$.
      \end{enumerate}

    \item[6.]  Try to determine how ends of edges at pairs of inverse vertices in the Heegaard
      diagram must be identified.
      \begin{enumerate}
      \item[A.] If there are generators which appear in the presentation with only one exponent,
        up to absolute value, and this exponent is large enough to cause nonuniqueness,
        report this fact and call \verb NonUnique_Initial_Diagram() .
      \item[B.] If there is a  ``valence-two-annulus" present, report this fact and call \\ 
        \verb NonUnique_Initial_Diagram() .
      \item[C.]  Otherwise, the identifications of the edges at pairs of inverse vertices are
        uniquely determined; compute what these identifications are, and continue.
      \end{enumerate}
    \item[7.]  Check whether the diagram just determined realizes the current presentation.
      \begin{enumerate}
      \item[A.]  If the diagram does not realize the current presentation, declare the initial
        presentation not realizable and return.
      \item[B.]  If the current presentation was obtained by eliminating  ``primitives" which were
        not  ``trivial generators", declare that the current presentation is realizable.
      \item[C.]   Otherwise, the current presentation was obtained from the original presentation
        by  ``level-transformations", and eliminating only  ``trivial generators". Declare
        the original initial presentation realizable.
        \end{enumerate}
      \end{enumerate}
\}

\question   It would be interesting to have an example of a presentation $P$ with the
    following properties.
   \begin{enumerate}
            \item     P is not realizable.
            \item      The program cannot detect that $P$ is not realizable.
             \item      There is a set $S_1$ of relators of $P$, each of which is a primitive, such that the
                        presentation $P_1$ obtained by forming the quotient of $P$ by the members of $S_1$, is
                        realizable, and $P_1$ uniquely determines a 3-manifold $M1 $.
               \item      There is a set $S_2$ of relators of $P$, each of which is a primitive, such that the
                        presentation $P_2$ obtained by forming the quotient of $P$ by the members of $S_2$, is
                        realizable, and $P_2$ uniquely determines a 3-manifold $M_2$.
                \item      $M_1$ and $M_2$ are not homeomorphic.                           
                
     \end{enumerate}                                   

                                        
  \section{THE DEFINITION OF  ``PAIRS OF SEPARATING VERTICES".}

        
                Suppose $P$ is a presentation which has minimal length, $RWG$ is the  ``reduced
        Whitehead graph" of $P$, and $RWG$ is connected. The definition of  ``pairs of separating
        vertices" has been chosen so that if $RWG$ has no pairs of separating vertices, then both
        RWG and the Whitehead graph $WG$ of $P$ have unique embeddings in the plane up to
        homeomorphism. Uniqueness of the embedding of $WG$ follows from uniqueness of the embedding
        of $RWG$, when no pairs of separating vertices exist, because all of the edges of $WG$, which
        join a given pair of vertices in $WG$, must be embedded so that they are parallel to each
        other, thus forming a  ``band" of parallel edges.
        
        \note   Because $RWG$ corresponds to a presentation which has minimal length, $RWG$ does not
        have any  ``cut vertices".
        
                Before giving the definition of  ``pairs of separating vertices", we need the following
        preliminary definitions.
        
        \definition     Let $V$ be a vertex of $RWG$, the  ``valence" of $V$ is the number of distinct vertices
                joined to $V$ by edges of $RWG$.
                
        \definition     A vertex $V$ in $RWG$ is a  ``major vertex" of $RWG$ if $V$ has valence greater than two
                in $RWG$. 
        
                Now suppose that $V1$ and $V2$ are two major vertices of $RWG$. Let $RWG'$ be the graph
        obtained by deleting $V1$ and $V2$ from $RWG$ and suppose that $RWG'$ has more than one
        component. We say that the pair $\{ V1,V2\}$ form a  ``pair of separating vertices" of RWG
        provided one of the following holds:
                \begin{enumerate}
                \item[1)]      RWG has only two major vertices, namely the pair $\{ $V1,V2$\}$, and:
                  \begin{enumerate}
                        \item[ a)]      $V1$ and hence $V2$, has valence greater than 3 or,
                        \item[b)]      There is more than one edge joining $V1$ and $V2$ in the Whitehead graph
                                WG of $P$.
                              \end{enumerate}
                 \item[2)]      $RWG'$ has more than two components, or $RWG'$ has two components, $V1$ and $V2$ are
                        joined by an edge and at least one component of $RWG'$ contains a major vertex
                        of $RWG$.
                        
                \item[3)]     $RWG'$ has two components, $V1$ and $V2$ are not joined by an edge and each component
                        of $RWG'$ contains a major vertex of $RWG$.
                        \end{enumerate}

\section{HOW THE PROGRAM DETERMINES THE IDENTIFICATIONS OF THE EDGES MEETING A VERTEX $V$ WITH THE EDGES MEETING THE INVERSE OF $V$ WHEN THE PROGRAM IS RECOVERING A HEEGAARD DIAGRAM FROM AN EMBEDDING OF THE WHITEHEAD GRAPH OF A PRESENTATION IN THE PLANE.}


                Suppose $P$ is a minimal length presentation with Whitehead graph $WG$, reduced
        Whitehead graph $RWG$, $RWG$ is connected, $RWG$ has no pairs of separating vertices, and
        P is realizable. Then $RWG$ has a unique embedding in the plane, and this unique embedding
        can be extended to a unique embedding of $WG$. Suppose this has been done. In order to
        recover the Heegaard diagram determined by $P$, the program needs to be able to determine,
        for each pair of inverse vertices, how the edges which meet one vertex are to be
        identified with the edges meeting its inverse. The problem falls naturally into two
        cases:
        \begin{enumerate}
        
                \item[1)]      The corresponding generator appears in $P$ only with exponents less than or
                        equal to two in absolute value.
                        
                \item[2)]      The corresponding generator appears in $P$ with an exponent which is greater
                        than two in absolute value.
                     
         \end{enumerate}
                We treat case 1) first. Suppose that $\{ Y,y\}$ is such a pair of vertices. Assume for
        the moment that there are no relators in $P$ which are powers of $Y$. Then among the 
        bands of parallel edges that join vertex $Y$ or vertex $y$ to other vertices of $WG$, let $M$
        be a band which contains a maximal number of edges. Let m be the number of edges of $WG$
        in $M$. Let $E$ be the total number of edges of $WG$ meeting $Y$ (which is of course equal to
        the total number of edges of $WG$ meeting $y$.) Note that if one of $\{ $Y$,y\}$ has valence two
        in $RWG$ and the other has valence greater than two in $RWG$, then we may choose the band
        M such that it meets the vertex of valence two. Next, note that it is a consequence of
        the minimal length of $P$ that $2m \leq E$.
                
                Suppose, for the purposes of illustration, that the band $M$ joins vertices $x$ and $Y$.
        The program then reads through the relators and their inverses, looking for appearances
        of subwords of length 3 which start with $XY$. Suppose it finds $n_1$ appearances of $XYF$,
        $n_2$ appearances of $XYg$ and say $n_3$ appearances of $XYh$. Then $m = n_1 + n_2 + n_3$, and the
        m edges meeting $Y$ must be identified with m consecutive edges meeting $y$ so that $n_1$ of
        these edges go to vertex $F$, $n_2$ of these edges goto vertex $g$ and $n_3$ of these edges go
        to vertex $h$. If either vertex $Y$ or vertex $y$ has valence greater than two in $RWG$, there
        is only one way to make such an identification. If both vertex $Y$ and vertex $y$ have
        valence two in $RWG$, then there are no more than two possible ways to make such an
        identification. (When there are two possible ways to make this identification, the
        program will later call its subroutine \verb Valence_Two() which will attempt to resolve the
        ambiguity.)
                
                This takes care of case 1) except when there are relators which are powers of $Y$.
        However, it is quite easy to make the appropriate modifications to handle this
        situation, so we will omit the details.
                
                Finally, we treat case 2). So suppose that $\{ Y,y\}$ is a pair of vertices such that
        the corresponding generator appears in $P$ with an exponent which is greater than two
        in absolute value, and suppose, for the moment, that there are no relators in P
        which are powers of $Y$. Since $Y$ appears in $P$ with an exponent which is greater then
        two in absolute value, there are edges in $WG$ joining $Y$ and $y$. Furthermore, all such
        edges are parallel to each other. This implies that $Y$ appears in $P$ with exponents
        which take at most three distinct absolute values, and these values are of the form
        $p, q$ and $p+q$, with $p$ and $q$ relatively prime. (See the lemma below for a proof of this.)
                
                In this case, the program counts the total number of appearances of $Y^p$, $Y^q$, and
        $Y^{p+q}$, in the relators and their inverses. Suppose the number of times $Y$ appears with
        each of these exponents is respectively $d$, $e$ and $f$ and at least two of these numbers
        are nonzero. Then proceeding counter-clockwise around vertex $Y$, starting with the first
        edge that joins $Y$ to a vertex distinct from $y$, one must meet either $d$ consecutive edges
        which represent $Y^p$ followed by $f$ consecutive edges which represent $Y^{p+q}$ and finally
        $e$ consecutive edges which represent $Y^q$ or one must meet $e$ consecutive edges which
        represent $Y^q$, followed by $f $ consecutive edges which represent $Y^{p+q}$ and finally d
        consecutive edges which represent $Y^p $.
                
                If one of $\{ Y,y\}$ has valence greater than two in $RWG$, the program can easily
        determine which of these two alternatives holds. Suppose, for example, that $Y$ has
        valence greater than two in $RWG$. And suppose that the first vertex following $y$ in
        counter-clockwise order about $Y$ is $x$ and that the vertex preceding $y$ in counter-
        clockwise order about $Y$ is $z$. Then $x$ and $z$ are distinct. The program again reads
        through the relators, this time looking for subwords of the form $XY^e$ and $ZY^e$ where $e$
        is one of $p, q, p+q $. The values of $e$ which appear give the program enough information
        to determine which of the two alternatives above is correct. It is then an easy matter,
        involving a simple formula, to determine explicitly what the appropriate identifications
        of the edges at $Y$ and $y$ must be.
                
                As in case 1), if both vertex $Y$ and vertex $y$ have valence two in $RWG$, then there
        are at most two possible ways to identify the edges at $Y$ with the edges at $y$. (As
        before, if this ambiguity arises, the program will later call its subroutine
        \verb Valence_Two() which will attempt to resolve the ambiguity.)
                
                This takes care of case 2) except when there are relators which are powers of $Y$,
        or $Y$ appears in the relators with only one exponent. Note that if there are relators
        which are powers of $Y$, then $Y$ must appear with only one exponent. And if $Y$ appears with
        with only one exponent, there is a genuine ambiguity.

                The next lemma extends a result due to Osborne for Heegaard diagrams of genus two
        to minimal length diagrams of arbitrary genus. It also underscores how obtaining
        diagrams without pairs of separating vertices simplifies the situation.
        
\lemma  Suppose $P$ is a minimal length realizable presentation, $X$ is a generator of $P$, $x$ is
        the inverse of $X$, the pair of vertices $\{ X,x\}$ do not form a pair of separating vertices
        of the reduced Whitehead graph $RWG$ of $P$, and $X$ appears in $P$ with an exponent $e$, where
        $|e| > 1$. Then, up to absolute value, $X$ can appear in $P$ with at most 3 distinct exponents,
        which must be of the form: $p,q$ and $p+q$, where $p$ and $q$ are relatively prime.
        
\pf   Let $D$ be a Heegaard diagram which realizes $P$, with $|D| = |P|$. Let $H$ be the under-
        lying handlebody. Since $X$ appears in $P$ with exponent e, where $|e| > 1$, there exist edges
        of $D$ joining vertex $X$ of $D$ to vertex $x$ of $D$. Since $\{ X,x\}$ do not form a pair of
        separating vertices of $RWG$, all of the edges of $D$ which join $X$ to $x$ must be parallel
        to each other. Let $N$ be a regular neighborhood in $D$ of $X$,x and an edge of $D$ joining $X$
        to $x$. Note that up to isotopy in $D$, we may assume that all of the edges of $D$ which join
        $X$ to $x$, lie in the interior of $N$.
                Note $\partial N$ cuts off a punctured torus from the Heegaard surface $\partial H$. The attaching
        curves of the 2-handles which realize $P$, are either disjoint from $\partial N$, or they are cut
        into sets of arcs by $\partial N$. It is well-known and not hard to show, that there can be at
        most 3 distinct sets of properly embedded arcs on a punctured torus, such that no two
        arcs in distinct sets are parallel. This implies that if a subarc of an attaching curve
        of a 2-handle lies in $N$, then it intersects $\partial X$ in one of at most 3 ways. But successive
        subarcs of an attaching curve, which lie in $N$, are separated by intersections of the
        attaching curve with cutting disks of $H$, from other handles, distinct from $X$. Thus the
        claim holds.    


\section{SOME REMARKS ABOUT OBSTRUCTIONS TO REALIZING A PRESENTATION UNIQUELY.}
                        
                
                Suppose that $P$ is a minimal length presentation. There are essentially three types of
        obstructions to uniqueness which arise as the program attempts to determine whether $P$ is
        realizable, each of which is fairly easy to understand.
                The most obvious obstruction occurs when the underlying handlebody $H$ contains an
        essential disk $X$, such that $\partial X$ is disjoint in $\partial H$ from the attaching curves of the 2-
        handles of a diagram.
                If $X$ separates $H$, the minimal length of $P$ implies that the associated 3-manifold $M$ is
        a connected sum. However, the program will often not be able to determine whether the sum
        is along a disk represented by $X$, or along a 2-sphere formed by $X$ and a disk lying in a
        3-ball used to cap-off a boundary component of $M$.
                If $X$ does not separate $H$, then again the associated 3-manifold is a connected sum,
        but the situation can be somewhat more complicated than the previous situation. There
        are essentially 3 possibilities.
                
                1)     $ \partial X$ is an essential separating curve in $\partial M$, so that $X$ represents a 1-handle which
                        joins two distinct boundary components of $M - N(X)$.
                
                2)      $\partial X$ is an essential nonseparating curve in $\partial M$, so that $X$ essentially represents a
                        disk connected sum of $M - N(X)$ and a solid torus.
                
                3)      $\partial X$ is capped off by a disk in a 3-ball used to cap-off a boundary component of $M$.
                        In this case, $X$ essentially represents an $S^1 \times S^2$ connected summand of $M$.
                        
                Often, the program will not be able to determine which of these possibilities is
        correct, and it should admit that it does not know. If the program does know how $M$ is
        constructed, then it should describe the boundary structure of $M$, and list the number
        of summands of the form $I \times D^2, S^1 \times S^2, S^1 \times D^2$ that are involved. The program does
        this in a somewhat cryptic way, however, which we will try and explain.
                What the program does which may be a little confusing, is it lumps all of these
         ``handles" together. For example, suppose $M$ has 3 summands, $M_1,M_2,M_3$. And suppose $M_1$ is
        closed, $M_2$ is a handlebody of genus 2, and $M_3$ is a handlebody of genus 3.
                The program will lump the handles of $M_2$ and $M_3$ together and tell you that there are
        5 $S^1 \times D^2$(s) present. However, it should also tell you that $M_1$ is closed, and that $M$ has
        2 boundary components, one of genus 2, and one of genus 3, this allows you to deduce that
        the 5 $S^1 \times D^2$(s) are partitioned into a handlebody of genus 2 and a handlebody of genus 3,
        and thus you have enough information to determine how $M$ is constructed.
                
                                
                     \section{PAIRS OF SEPARATING VERTICES IN THE REDUCED WHITEHEAD GRAPH OF $P$.}
                        
                                
                The second type of obstruction, and the one that seems to arise most frequently in
        practice, is the existence of pairs of separating vertices in the reduced Whitehead
        graph of $P$. This topic is discussed in several other places, so we will not say any more
        about it here, and instead pass on to the third type of obstruction.
                        
                                
        \section{``ANNULI" AS OBSTRUCTIONS TO REALIZING A PRESENTATION UNIQUELY.}
        
        
                The third type of obstruction, is perhaps the most interesting from a topological
        point of view, because, when it arises, it tends to give some topological information
        about the structure of the 3-manifold $M$, which might not otherwise be easily detected.
                This obstruction arises when there is an essential annulus $A$ in the underlying
        handlebody $H$ such that $\partial A$ is disjoint in $\partial H$ from the attaching curves of the 2-handles
        of a diagram which realizes $P$. We note that if such an annulus exists, then $A$ may or
        may not represent an essential annulus, disk or 2-sphere in $M$ depending upon whether
        $A$ is incompressible in $M$ and whether one or both boundary components of $A$ bound disks
        in $M$.
                At the present time, the program alerts you to the fact that it has found such an
        annulus in the underlying handlebody, but it then leaves further disposition of the
        situation up to you.                    


                As far as what characterizes uniqueness of realizability in general, we have
        considered the question, and believe we can show that the following result is true.
        
        \theorem        Suppose $P$ is a minimal length presentation on $g$ generators, $g > 1$, such
        that the reduced Whitehead graph of $P$ is connected, and suppose that $D$ and $D'$ are
        distinct Heegaard diagrams of genus $g$ with $|D| = |D'| = |P|$, such that both $D$ and $D'$
        realize $P$. Let $H$ be the underlying handlebody of $D$. Then there is an essential annulus
        $A$ in $H$, with $\partial A$ disjoint in $\partial H$ from the attaching curves of the 2-handles of $D$.
        Similarly, there exists such an annulus in the underlying handlebody $H'$ of $D'$.
        
                The proof is too long to give here, but we can indicate the general ideas.
        Suppose that $P'$ is a minimal length presentation obtained from $P$ via a sequence of
        level-transformations. If the reduced Whitehead graph of $P'$ embeds uniquely in the
        plane, then either there exists a unique diagram $D'$, with $|D'| = |P'|$, which realizes
        $P'$, or a Valence-Two-Annulus exists, or there is a generator $X$ of $P'$ which appears
        with only one exponent, which is greater than two in absolute value, and again an
        annulus exists. Hence every presentation obtainable from $P$ via level-transformations,
        has a reduced Whitehead graph which has pairs of separating vertices. In particular
        the reduced Whitehead graph of $P$ has pairs of separating vertices.
                Next, we use, what are essentially covering space techniques, to show that there
        exists an essential singular annulus $A$, in the underlying handlebody $H$, with $\partial A$
        disjoint from the attaching curves of the 2-handles of $D$. One can then do surgery on
        $A$ in order to obtain the desired essential annulus.
                 

 \section{HOW THE PROGRAM HANDLES PRESENTATIONS WHICH CONTAIN DUPLICATED RELATORS.}
                        
                The program will accept and handle presentations which contain relators which are
        identical up to taking cyclic conjugates or inverses, but it is not very tolerant of
        them. The reason for this is not hard to see, and comes from the fact that the program
        places a heavy reliance on the uniqueness of the realization of a minimal length
        presentation in order to insure that it is preserving the homeomorphism type of the
        associated 3-manifold.
                Suppose, for example, that $P$ is a presentation which contains two relators $R_1$ and
        $R_2$, which are identical up to taking cyclic conjugates and inverses. Let $H$ be the
        underlying handlebody and suppose $D$ is a realization of $P$ with $|D| = |P|$, and suppose
        that $R_1$ and $R_2$ are realized by simple closed curves $C_1$ and $C_2$ in $\partial H$, such that $C_1$ and
        $C_2$ are not parallel in $\partial H$. Then $D$ is clearly not unique, since $P$ obviously has another
        realization in which $C_1$ and $C_2$ are embedded in $\partial H$ such that they are parallel.
                Thus the situation is the following: Either $P$ does not have a unique realization,
        in which case, the program will make no further use of $P$, or $P$ has a unique realization,
        in which case, copies of duplicated relators are redundant, and the program can and will
        delete the extraneous copies.           

\section{ AN EXCEPTION TO THE RULE THAT PRESENTATIONS IN WHICH A GENERATOR APPEARS WITH ONLY ONE  EXPONENT, WHICH IS GREATER THAN TWO IN ABSOLUTE VALUE, ARE NOT UNIQUELY REALIZABLE.}
        
                Generally, if a generator appears in the relators with only one exponent, and that
        exponent is greater than two in absolute value, then the Heegaard diagram is not unique.
        There are exceptions to this however. These exceptions occur when the  ``reduced" Whitehead
        graph is homeomorphic to a circle and there are not too many edges in the Heegaard diagram.
        The program contains a routine which is called to check whether uniqueness holds in this
        special set of circumstances. Without this special check, the program would declare that
        the diagram corresponding to the relator $AAABB$ was not unique because generator $A$ appears
        only with exponent 3. However this diagram is unique because there is an involution of the
        diagram which exchanges the two major faces of the diagram. Here is an outline of this
        test for uniqueness.
        
        \verb Test_Uniqueness() \{
        \begin{enumerate}
                \item     If more than one generator appears with maximal exponent greater than two in
                        absolute value, declare the diagram not unique and return.
                \item      If the generator which appears with maximal exponent greater than two in absolute
                        value appears with exponent $e$ and $e$ is not 3,4 or 6, declare the diagram not
                        unique and return. Note that 3,4 and 6 are the only integers greater than two
                        which have the property that at most two smaller integers are relatively prime
                        to them.        
                \item      If there is a vertex of the reduced Whitehead graph which is joined to more than
                        two other vertices, declare the diagram not unique and return.
                \item      If $i$ and j are vertices of the Whitehead graph, which are not inverses, and $i$ and
                        j are joined by more than one edge in the Whitehead graph, declare the diagram
                        not unique and return.
                \item      Declare the diagram unique and return.  
        \end{enumerate}
        \}
        
        This routine has been involved when the program produces messages like the following:
        \begin{enumerate}        
               \item       ``The diagram is not unique because there is a generator which appears with
                        only one exponent and that exponent is greater than 6."
                        
                \item       ``The diagram is not unique because there is a generator which appears only
                        with exponent 5."
                        
                \item       ``The diagram is not unique because there is a generator which appears only with
                        exponent 3 or only with exponent 4 and this exponent occurs more than once."
                        
                \item ``The diagram is not unique because there is a generator which appears with only
                        one exponent, either 3,4 or 6, and a needed symmetry does not exist."
        \end{enumerate}

\section{HOW THE PROGRAM ATTEMPTS TO FIND PRESENTATIONS, WHICH HAVE NO PAIRS OF SEPARATING VERTICES IN THEIR REDUCED WHITEHEAD GRAPHS, BY PERFORMING LEVEL-TRANSFORMATIONS.}

        
                The program puts a premium on obtaining minimal length presentations which have no
        pairs of separating vertices in their reduced Whitehead graphs. It does this since such
        graphs have a unique embedding in the plane, and this unique embedding can be extended
        to a unique embedding of the Whitehead graph of the presentation in the plane. And this,
        in turn, makes finding the Heegaard diagram which realizes the presentation fairly easy.
                
                Suppose $P$ is a minimal length presentation whose reduced Whitehead graph $G$ is
        connected and $\{ X, Y\}$ is a pair of separating vertices of $G$. Let $x$ denote the inverse of $X$,
        and let $y$ denote the inverse of $Y$. We need to distinguish between two types of separating
        vertices, and so we make the following definition:
        
                \definition     Let $G'$ be the graph obtained by deleting $X$ and $Y$ from $G$. Say $\{ X, Y\}$ is a
                 ``Type 1" pair if $Y$ = $x$, or if there exists a component $C$ of $G'$ which contains
                both $x$ and $y$. If no such component $C$ exists, say that $\{ X, Y\}$ is a  ``Type 2" pair.
        
                Now, because $P$ has minimal length, the following lemmas hold.
                
                \lemma  Let the notation be as above. Suppose that $\{ X, Y\}$ is a Type 1 pair of
                separating vertices of $G$. Let $C$ be a component of $G'$ such that neither $x$ or $y$ lies
                in $C$. Let m be the number of edges of the Whitehead graph $WG$ of $P$ which join
                vertices of $C$ to vertex $X$. Similarly, let $n$ be the number of edges of $WG$ which
                join vertices of $C$ to vertex $Y$. Then $m = n$.
                
                \lemma  Let the notation be as above. Suppose that $\{ X, Y\}$ is a Type 2 pair of
                separating vertices of $G$. Then $G'$ has only two components, and there is no edge
                joining vertex $X$ to vertex $Y$ in $G$.
        
                \lemma  Let the notation be as above. Suppose that $\{ X, Y\}$ is a Type 2 pair of
                separating vertices of $G$. Let $C_x$ be the component of $G'$ which contains vertex $x$,
                and let $C_y$ be the component of $G'$ which contains vertex $y$. Let m be the number of
                edges of the Whitehead graph $WG$ of $P$ which join vertices of $C_y$ to vertex $X$.
                Similarly, let $n$ be the number of edges of $WG$ which join vertices of $C_x$ to vertex $Y$.
                Then $m = n$.
                
                These lemmas imply that if $P$ has minimal length, and the reduced Whitehead graph $G$
        of $P$ has a pair of separating vertices, then there always exist level-transformations.
        In particular, there always exists level-transformations which can be viewed as  ``sliding"
        around a subgraph of $G$, which is attached to the rest of $G$ at only two vertices. However,
        as indicated by the lemmas, the level-transformations involved differ according to whether
        the pair of separating vertices is a Type 1 or Type 2 pair. The program includes routines
        designed to deal with each of these two cases. The program's routine \verb Level_Transformations() deals with Type 1 pairs, and the program's routine \verb Level_Transformations_2() deals with Type 2 pairs.
                
                Here is a brief description of how these routines work. \verb Level_Transformations() and \\ 
        \verb Level_Transformations_2() call each other recursively, with \verb Level_Transformations() \ \\
        being
        called first. Each invocation of \verb Level_Transformations()  ``owns" a presentation. On entry,
        the routine first checks that the presentation being passed to it is distinct from those
        passed to all previous invocations. The routine also checks at this point whether the
        previous invocation of \verb Level_Transformations() has managed to get rid of all pairs of
        separating vertices in the graph. If so, the routine saves this presentation and returns.
        Otherwise this presentation  ``belongs" to this invocation, and the routine saves a copy of
        the presentation.
                
                The routine then looks for a pair of separating vertices. If it finds a pair, say
        $\{ X, Y\}$, then it looks for a component  ``TheComp", of the separation produced by deleting
        $\{ X, Y\}$, with the property that neither $X$ inverse or $Y$ inverse are in TheComp.
                
                Then, the objective is to  ``slide" the component TheComp around the Whitehead graph,
        sliding first to the  ``left" and then (if necessary) to the  ``right" until either TheComp
        is attached to the main body of the graph at more than two distinct vertices, or we
        arrive at a vertex of attachment $V$ of TheComp with the property that $V$ inverse is a
        member of TheComp, or we manage to slide TheComp completely around the graph along some
        path, and the presentation has returned to the original presentation. If the last case
        occurs, then there is an  ``annulus" present. Otherwise we call \verb Level_Transformations() \ 
        recursively and pass it the new presentation. (See below for more precise details about
        what it means to  ``slide" a component around.)
                
                Finally, when \verb Level_Transformations() has done all it can with the presentation
        passed to it, \verb Level_Transformations() calls \verb Level_Transformations_2(), which deals with
        any Type 2 pairs. Except for differences in the sets of vertices which it tries to slide
        around, \\  \verb Level_Transformations_2() is similar to \verb Level_Transformations(). Outlines of
        these two routines are given below. In particular, the outline of \verb Level_Transformations_2() indicates how the sets of vertices which it tries to slide around differ from the
        sets of vertices that \\ 
        \verb Level_Transformations() slides around.
        
\verb Level_Transformations(P) \{ 
\begin{enumerate}
  
        \item[1)]      Check whether presentation $P$  ``belongs" to a previous invocation of \\ 
                \verb Level_Transformations(). If it does, return.    
        \item[2)]      Check whether $P$ has any pairs of separating vertices.
          \begin{enumerate}
                \item[A)]      If $P$ has no pairs of separating vertices, see if $P$ is  ``new"
                  \begin{enumerate}
                        \item[a)] If $P$ is  ``new", save a copy of $P$ and return.
                        \item[b)] If the program already has a copy of $P$ on file, return.
                        \end{enumerate}
                \item[B)]      P has pairs of separating vertices, so continue.
                \end{enumerate}
        \item[3)]      Save a copy of $P$. $P$ now  ``belongs" to this invocation of \verb Level_Transformations() .             
        \item[4)]      Find the components produced by deleting the current set of separating vertices from
                the reduced Whitehead graph of $P$.
        \item[5)]      If $\{ X, Y\}$ are the current separating vertices, look for a component  ``TheComp" of the
                separation, with the property that neither $x$ or $y$ lie in TheComp.
                \begin{enumerate}
                \item[A)]      Slide TheComp over as many handles as possible; starting with handle $X$. Let P'
                        be the presentation obtained from $P$, by this slide of TheComp.
                \item[B)]      If there is an annulus present, return.
                \item[C)]      Call \verb Level_Transformations( $P'$) . 
                \item[D)]      Restore $P$.
                \item[E)]      Slide TheComp over as many handles as possible; starting with handle $Y$. Let $P''$
                        be the presentation obtained from $P$, by this slide of TheComp.
                \item[F)]      If there is an annulus present, return.
                \item[G)]      Call \verb Level_Transformations( $P''$).
                \item[H)]      Restore $P$.
                \item[I)]      If there are other components of the separation which we can slide around,
                        go to 5).
                      \end{enumerate}
            \item[6)]      Look for another pair of separating vertices.
              \begin{enumerate}
                \item[A)]      If there are other pairs of separating vertices, go to 4).
                \item[B)]     All pairs of separating vertices have been checked for Type 1 separations. Call \\ 
                        \verb Level_Transformations_2( $P$ ).
                \item[C)]      Return.
              \end{enumerate}
\end{enumerate}
\}
                
\verb Level_Transformations_2(P) \{
\begin{enumerate}
        \item[1)]      Find a pair of separating vertices of the reduced Whitehead graph $G$ of $P$.
        \item[2)]      If the current pair of separating vertices $\{ X, Y\}$, is a Type 2 pair, do the following:
          \begin{enumerate}
                \item[A)]      Let $G'$ be the graph obtained by deleting $X$ and $Y$ from $G$. Let $C_x$ be the component
                        of $G'$ which contains vertex $x$, and let $C_y$ be the component of $G'$ which contains
                        vertex $y$. Let CXx be $C_x$ union vertex $X$. Let CYy be $C_y$ union vertex $Y$.

                     
                \item[B)]      Slide CXx over as many handles as possible; starting with handle $Y$. Let P'
                        be the presentation obtained from $P$, by this slide of CXx.
                \item[C)]      If there is an annulus present, return.
                \item[D)]      Call \verb Level_Transformations( $P'$) .
                \item[E)]      Restore $P$.
                \item[F)]      Slide CYy over as many handles as possible; starting with handle $X$. Let P''
                        be the presentation obtained from $P$, by this slide of CYy.    
                \item[G)]      If there is an annulus present, return.
                \item[H)]      Call \verb Level_Transformations( $P''$).
                \item[I)]      Restore $P$.
                \end{enumerate}
        \item[3)]      Look for another pair of separating vertices in the reduced Whitehead graph of $P$.
          \begin{enumerate}
                \item[A)]      If there are other pairs of separating vertices, go to 2).

                \item[B)]      If all pairs of separating vertices have been checked, return.
                \end{enumerate}
                \end{enumerate}
                \}

                        
\section{A MORE DETAILED EXPLANATION OF WHAT IT MEANS TO ``SLIDE" A COMPONENT OF A SEPARATION AROUND.}
                                
                Suppose that $\{ X, Y\}$ are a pair of separating vertices of $RWG$, and  ``TheComp" is a
        component of the separation. Here are is a more precise description of what the program
        does when it  ``slides" TheComp around. Suppose that $P$ has been realized by a Heegaard
        diagram $D$, with $|D| = |P|$. Let $n > 1$, be the smallest integer such that there exists a
        set $SX$of oriented paths in $D$, which have the following properties:
        \begin{enumerate}
                \item[1)]      Each path in $SX$is a union of oriented edges of $D$ with the head end of edge $i$
                        identified in $D$ with the tail end of edge $i+1$ for $ 1 \leq i <  n$.
                \item[2)]      The first edge of each path in $SX$has its tail end at a vertex in TheComp and
                        its head end at vertex $X$.
                \item[3)]      If 1 < $i$ < $n$, the ith edges of all paths in $SX$ join the same pair of vertices of
                        $RWG$.
                \item[4)]      If $i$ = $n$, either a), b) or c) below holds:
                  \begin{enumerate}
                        \item[a)]      The $n$th edges of two distinct paths in $SX$ have their head ends at distinct
                                vertices of $RWG$ and c) has not occurred.
                        \item[b)]      The head end of the $n$th edge of each path in $SX$ lies at a vertex $V$ of $RWG$,
                                such that the vertex of $RWG$ which represents the inverse of $V$, lies in
                                TheComp.
                        \item[c)]      The tail end of the $n$th edge of each path in $SX$ lies at vertex $Y$, and the
                                head end of the $n$th edge of each path in $SX$ lies at a vertex in TheComp.
                  \end{enumerate}
                  \end{enumerate}
                Now there is a sequence of $n-1$ level-transformations, which topologically amounts
        to sliding TheComp over handle $X$ and continuing until:
        \begin{enumerate}
                \item[1)]      If a) holds TheComp is now attached to rest of $RWG$ at more than two distinct
                        vertices.
                \item[2)]      If b) holds TheComp is still attached to the rest of $RWG$ at only two distinct
                        vertices, but the vertices of attachment form a Type 2 pair of separating
                        vertices.
                \item[3)]      If c) holds the underlying handlebody $H$ contains an essential annulus $A$, with
                        $\partial A$ disjoint from the attaching curves of the 2-handles which realize $P$.
        \end{enumerate}
                                        

\remark We mention that these routines, while they seem to work quite well in practice, are
        only a means by which to search for a presentation which has no pairs of separating
        vertices in its reduced Whitehead graph, or failing that, to locate an essential annulus
        in the underlying handlebody. In particular, they do not provide an algorithm to either
        find such a presentation or to find an annulus. Thus answers to the following questions
        would be helpful in either placing these routines on a firmer footing, or in suggesting
        alternatives.

QUESTION 1)     What characterizes the set of minimal length, realizable, presentations which
        have the property that some presentation obtainable from them via level-transformations,
        has a reduced Whitehead graph without any pairs of separating vertices ? In particular,
        is the answer to Question 2) yes ?

QUESTION 2) Suppose $P$ is realizable, $P$ has minimal length, and there is no essential annulus
        $A$, in the underlying handlebody $H$, such that $\partial A$ is disjoint from the attaching curves of
        the 2-handles of $P$. Suppose the reduced Whitehead graph of $P$ has a pair of separating
        vertices, does there exist a presentation $P'$, obtainable from $P$ via a sequence of level-
        transformations, such that the reduced Whitehead graph of $P'$ has no pairs of separating
        vertices ?
        
QUESTION 3)     Suppose $P$ is a realizable minimal length presentation such that some presentation
        $P' $, obtainable from $P$ via level-transformations, has a reduced Whitehead graph without
        pairs of separating vertices. What is the most efficient way to find $P'$ ?
        
                
                        
\section{LEVEL TRANSFORMATIONS OF PRESENTATIONS (General case)}

                Let $P$ be a minimal length presentation whose Whitehead graph $WG$, is connected.
        Here is a short description of an efficient procedure we have devised, which uses
        network flows to find all of the nontrivial level-transformations of $P$. We apologize
        for being fairly sketchy about many of the details, and for leaving most of the proofs
        to the reader.
                
                Suppose that $A$ is a vertex of $WG$, and a is the vertex of $WG$ which represents the
        inverse of $A$. We consider the problem of finding all nontrivial level-transformations,
        in which generator $A$ acts on the remaining generators of $P$. By the preceding results,
        this is equivalent to finding all of the minimal cut-sets of edges of $WG$, which
        separate vertex $A$, from vertex $a$. Let $C$ be such a minimal cut-set. We can look for $C$
        in following way. Consider vertex $A$ as the source vertex and vertex a as the sink
        vertex of a flow in $WG$, and then find a maximal flow $F$ in the network $WG$ from vertex $A$
        to vertex $a$. Interpret $F$ as a maximal collection of directed paths from vertex $A$ to
        vertex $a$, such that no two paths in $F$ have any edges in common.

                Let $G$ be a network all of whose edges are undirected. Note that if $F$ is a flow on
        $G$, then we may view those edges of $G$, which are traversed by paths in $F$, as assuming an
        induced direction. Note this is well defined, since an edge of $G$ is traversed by at
        most one directed path in $F$.
        
        \definition Suppose that $X$ and $Y$ are a pair of vertices of $G$, which are joined by $n$ edges of
        $G$. Say that the set of edges joining $X$ and $Y$ is  ``saturated" if all $n$ of these edges
        have the same induced direction, and otherwise say the set of edges joining $X$ and $Y$
        is  ``unsaturated".
                
                The objective is to produce a sequence of new networks by identifying vertices in
        WG, and new maximal flows on these networks, until finally, we obtain a network $G$ and
        a maximal flow $F$ on $G$, such that $F$ is acyclic and each set of edges, which joins a pair
        of vertices of $G$, is saturated. The point is that these changes can all be made without
        affecting the set of minimal cut-sets that separate source and sink. And, the set of
        minimal cut-sets can be easily enumerated, when the flow $F$ is acyclic, and all bands of
        edges, of the network, are saturated.
        
                Let $G$ be a network, and let $F$ be a flow on $G$. Observe that if $X$ and $Y$ are vertices
        of $G$ and some of the edges of joining $X$ and $Y$ are directed from $X$ to $Y$ and some of the
        edges joining $X$ and $Y$, are directed from $Y$ to $X$, then we can do  ``surgery" on the paths
        in $F$ to obtain another flow $F'$ such that any two directed edges joining $X$ and $Y$ have the
        same induced direction, and $F'$ has the same number of paths joining source and sink as
        $F$ does.
                        
                Suppose this has been done for all pairs of vertices of $G$. Thus, we may assume that
        if $X$ and $Y$ are vertices of $G$, which are joined by edges of $G$, then any two directed
        edges joining $X$ and $Y$ have the same induced direction.
                
                The following lemma is an immediate consequence of minimality and the fact that a
        maximal flow contains $V(A)$ independent edge-wise disjoint paths.
        
        \lemma  Suppose $X$ and $Y$ are vertices of $G$, which are joined by an unsaturated set of
        edges in $G$, then $X$ and $Y$ lie on the same side of any minimal cut-set which separates
        source from sink.
                        
                Suppose $X$ and $Y$ are joined by an unsaturated set of edges, then we can form a new graph
        $G'$, from $G$, by identifying $X$ and $Y$ and deleting any edges which joined $X$ to $Y$. We do
        this operation repeatedly, and if necessary, perform surgery on the flow $F$, so that all
        sets of edges joining pairs of vertices of $G'$ are saturated.
        
                Next, suppose that $G$ is a network, $F$ is a maximal flow on $G$, from source to sink,
        and $F$ has a cycle. Then we can do surgery on $F$ and obtain a new maximal flow on $G$ from
        source to sink, such that $F'$ has the property that the sets of edges in the cycle are
        unsaturated. This allows us to identify all of the vertices of the cycle to a single
        vertex, to delete all of the edges of $G$, which join vertices in the cycle to each other,
        and to further contract the graph. It is also not hard to find another maximal flow on
        the contracted graph.
        
        \definition     Suppose $G$ is an acyclic directed graph. Let $X$ and $Y$ be vertices of $G$. Say $Y$ is
        a  ``descendant" of $X$, if there is a directed path in $G$ from $X$ to $Y$.
        
        \definition     Let $S$ be a subset of the vertices of $G$. $S$ is  ``hereditarily closed" if whenever
        a vertex $X$ lies in $S$, then each descendant of $X$ also lies in $S$.
        
                Each hereditarily closed subset $S$ of the vertices of $G$ partitions the vertices of $G$
        into two subsets, $S$ and its complement. This partition in turn induces a partition of the
        vertices of $WG$ into two subsets, and the edges of $WG$ which join vertices in one subset to
        vertices in the other subset, form a minimal cut-set of $WG$. These minimal cut-sets give
        us the desired set of level-transformations.
        
                
        
\section{THE STRATEGY THE PROGRAM USES TO OBTAIN NEW PRESENTATIONS BY FORMING BANDSUMS.}
        
                Suppose $P$ is a minimal length presentation whose Whitehead graph is connected, and
        $D$ is a Heegaard diagram with $|D| = |P|$ which realizes $P$. Let $H$ be the underlying 
        handlebody. The relators in $D$, intersect the boundaries of the cutting disks of $H$ in        
        points, which cut the boundaries of these cutting disks into arcs. When the program
        forms bandsums, it uses subarcs of the boundaries of the cutting disks as the  ``band"
        along which the sum is formed. In a typical diagram, many of these arcs are  ``parallel"
        and hence lead to the same bandsum. The program therefore restricts its attention to
        bandsums formed along those subarcs, of the boundaries of cutting disks of $H$, which
        correspond to  ``corners" of the faces of the reduced Whitehead graph of $P$.
        
                When the program forms bandsums, it is in one of two modes. In the GoingDown mode,
        the program is attempting to reduce the length of a diagram by performing automorphisms
        and bandsums. So, in this mode, the program checks bands corresponding to all of the
        corners of the faces of the reduced Whitehead graph of $P$, and it looks for a bandsum
        which reduces the length of $P$ as much as possible.

                Otherwise, the program is in its GoingUp mode, and it looks for bandsums which
        increase the length of $P$. This mode is an exploratory mode, and so the program needs to
        be able to generate a spectrum of presentations. Here is a brief description of how it
        currently does this.

                Each presentation, which the program saves, can be used by the program as the root
        of a new branch of the program's search tree. The program will try to grow a new branch
        of the tree from a given presentation up to $2g$ times, provided the presentation is a
        presentation on $g$ generators. Suppose $n$ is an integer with $0 leq n < 2g$, and this is the
        $n$th attempt by the program to grow a new branch of the program's search tree from
        presentation $P$.

                The program selects, at random, $3(2g - n)$ edges, as above, which will serve as trial
        bands. Then the program finds, in this set of trial bands, a band which yields a bandsum
        that increases the length of $P$ by the smallest nonnegative amount, and performs that
        bandsum, yielding a new presentation $P'$. The program then obtains the diagram of $P'$,
        and returns to form another bandsum, again selecting a subset of $3(2g - n)$ random trial
        bands.
                This process is continued until one of the following occurs.
                \begin{enumerate}
                  \item[1)]      There exists an automorphism which reduces the length of the new presentation $P'$.
                
                \item[2)]      The lengths, of the new presentations produced, are more than double the length of
                        the first presentation, on which the program formed the original bandsum.
                
                \item[3)]      The number of bandsums formed, without anything interesting happening, is greater
                        than $4g$.
                        
                \item[4)]      The program is unable to find the diagram corresponding to $P'$.        
                \end{enumerate}
                If case 1) occurs, the program reduces $P'$ to minimal length, saves a copy of $P'$ as
        the \\  ``\verb Top_Of_Chain ", and switches to GoingDown mode. After returning from GoingDown mode,
        the program replaces the original presentation $P$ with the copy of $P'$, saved as
        \verb Top_Of_Chain , and the program resumes forming bandsums, starting from this presentation.
        This process is continued until terminated by cases 2), 3) or 4) occurring, or the
        program has generated and explored a sequence of more than 5 \verb Top_Of_Chain presentations.

                Finally, when the program exits this loop, $n$ is incremented, so the next search will
        be over a smaller set of trial bands.   
                        
        
\section{HOW THE PROGRAM ATTEMPTS TO DETERMINE THE IDENTIFICATIONS OF  EDGES OF THE HEEGAARD DIAGRAM AT PAIRS OF VERTICES, EACH OF  WHICH HAVE VALENCE TWO IN THE REDUCED WHITEHEAD GRAPH.}
        
        
                Here is a brief description of the routine which attempts to resolve ambiguities
        about how edges should be identified, in the Heegaard diagram, for a pair of vertices
        both of which have valence two in the reduced Whitehead graph of a presentation.
        
                Suppose $P$ is a minimal length presentation, with Whitehead graph $WG$, and reduced
        Whitehead graph $RWG$. Suppose that $RWG$ has no pairs of separating vertices, so that RWG
        and $WG$ have unique embeddings in the plane. Suppose that both vertex $X$ of $RWG$ and the
        the inverse vertex $x$ of $X$ in $RWG$, have valence two in $RWG$. And, suppose that the program
        has determined that there are two possible ways to identify the edges of $WG$, which meet
        $X$ with the edges of $WG$, which meet $x$. Let $B$ be one of the bands of parallel edges in $WG$,
        which meet vertex $X$, and suppose the edges in $B$ do not join vertex $X$ to vertex $x$. We note
        that, since there is an ambiguity about how these identifications should be made, $B$
        contains at least two edges.
                The program has computed the two possible ways that the edges meeting vertex $X$ can
        be identified with the edges meeting vertex $x$, and has stored these possibilities in two
        arrays, which we will call id1[] and id2[]. Next, the program assumes that the
        identification in id1[] is correct. Using the information in id1[], the program can 
        trace the left-hand edge of $B$, and the right-hand edge of $B$, backward over vertex $X$, and
        it can follow the paths taken by the left-hand and right-hand edges of $B$ until they leave
        vertex $x$, and split apart, with each path going to a distinct vertex of $WG$. (If this
         ``splitting" did not occur, then there would not be any ambiguity about how edges should
        be identified at $X$ and $x$.)
                Suppose the splitting takes place so that one path leaves vertex $x$ and goes to vertex
        a of $WG$, while the other path leaves vertex $x$ and goes to vertex b of $WG$. The program now
        reverses direction, and traces these paths in the opposite direction, starting from vertex
        a in one case, and starting from vertex $b$, in the other. The program continues to trace
        these two paths through the diagram, using known information about identifications at each
        new vertex reached, until one of the following occurs:
                \begin{enumerate}
                \item[1)]      The two paths traced out have each become longer than the longest relator in $P$. In
                        this case, $P$ is not realizable, and the program reports that fact.
                
                \item[2)]      The two paths diverge, and proceed to distinct vertices of $WG$.
                
                \item[3)]      The two paths have reached a vertex say $Y$, which is one of a pair of vertices of
                        valence two in $RWG$, and the identification of the edges of $WG$ meeting $Y$ with the
                        edges of $WG$ meeting $y$ is still ambiguous.
                        
                      \end{enumerate}

                Suppose 2) occurs, and suppose that the path which originated at vertex $a$ goes to
        vertex $C$ of $WG$, while the path that originated at vertex b goes to vertex $D$ of $WG$. Thus,
        if $P$ is realizable, and the identification in id1[] is correct for the pair $\{ X,x\}$, then
        subwords of the form $AWC$, and BWD must occur in the relators of $P$, where $W$ is the  ``word"
        traced out by the two paths up to the point where they diverged. While, if the
        identification in id2[] is correct for the pair $\{ X,x\}$, then        subwords of the form BWC, and
        AWD must occur in the relators of $P$, where $W$ is the  ``word" traced out by the two paths up
        to the point where they diverged.
                Now it is not hard to show that, under these circumstances, if $P$ is realizable, then
        at least two and at most 3 of the the 4 subwords  $\{ AWC, BWD, AWD, BWC\}$ can occur in the
        relators of $P$. Thus, by scanning the relators in $P$, the program can determine which of
        these possibilities occurs, and then it can determine which identification, of the edges
        meeting $X$ with the edges meeting $x$, is correct. This reduces the number of pairs of
        vertices with ambiguous identifications, and the program can now use this information to
        try to resolve ambiguous identifications of other pairs of vertices.


                                                                \section{VALENCE-TWO-ANNULI IN DIAGRAMS}
                                                                
                If case 3) occurs, the program marks the band $B$ as having been traced, and looks for
        another band of edges which it has not traced. If the program cannot find any band of
        edges, which it has not traced, and which leads to case 1), or case 2), then the diagram
        contains a cycle of valence two vertices, which form a  ``Valence-Two-Annulus". In this
        situation, there is an annulus $A$ in the underlying handlebody $H$, such that: $\partial A$ swallows
        an ambiguous vertex of valence two, then $\partial A$ follows the left and right hand edges of the
        band of parallel edges, which leave this vertex, to another ambiguous vertex of valence
        two, swallows that vertex, and continues, in this manner, until it returns to the original
        vertex of valence two.
                If a Valence-Two-Annulus exists, then the program can arbitrarily choose either of
        the two possible ways to identify the edges meeting one pair of valence-two vertices,
        provided at least one vertex of the pair lies in the cycle which constitutes the annulus,
        and then this choice determines the identifications of all other pairs of ambiguous
        valence-two vertices, for which at least one vertex of the pair lies in the cycle. By
        repeating this step, the program can eventually determine the identifications of all
        ambiguous pairs of valence-two vertices.
                There are a couple of degenerate situations which can occur. One possibility is that
        all of the vertices in $RWG$ are of valence-two, and the annulus in question has swallowed
        everything. In this case, we can ignore the annulus and proceed. Another possibility is
        that the annulus in question has swallowed all but two vertices of $RWG$. In this case, the
        omitted vertices form a pair of inverse vertices, and again the annulus does not cause
        a problem, so we can ignore it and proceed. (Observe that in this case, $\partial A$ represents a
        power of a free generator of $P$.)
        

\section{SOME RESULTS ABOUT REALIZING WORDS BY  ``PARALLEL PATHS" AND THE EXISTENCE OF  ``ANNULI".}


                Suppose $P$ is a presentation which has minimal length, and $W$ is a word such that $W$
        makes $n$ pairwise disjoint appearances in the relators of $P$ and their inverses. Let $D$ be
        a Heegaard diagram, with $|D| = |P|$, which realizes $P$. It will be useful to have some
        results which indicate under what conditions the pairwise disjoint subarcs which
         ``realize" the $n$ appearances of $W$ in the relators, must all be  ``parallel" to each other
        in $D$. We start with the following definition of  ``parallel edges":
        
        \definition     Suppose that $X$ and $Y$ are two vertices of a Heegaard diagram $D$, and there are $n$
        edges joining vertex $X$ to vertex $Y$ in $D$, where $n$ is greater than one. We will say that
        these $n$ edges are  ``parallel edges" of $D$ if there exists a pair $E1$ and $E2$ of these edges,
        such that one of the two complementary regions of the 2-gon, formed by $E1$  and $E2$ together
        with $X$ and $Y$, contains the other $n-2$ edges joining $X$ and $Y$ and this complementary region
        also contains no vertices of $D$.
        
                If $W$ is a word of length $K+1$, and $w$ is an oriented arc that  ``realizes" $W$ in $D$, then
        $w$ is cut into a set of $K$ successive oriented edges by the vertices of $D$. If $W$ appears n
        times in the relators, where $n$ is greater than one, and we want to say that the $n$ arcs
        which  ``realize" $W$ are  ``parallel", then we certainly want each of the $K$ sets of edges,
        into which the $n$ arcs that realize $W$ are cut by the vertices, to be sets of parallel
        edges. However, it is not enough for each of these $K$ sets of $n$ edges each to be sets of
        parallel edges, we also want successive sets of parallel edges to  ``match up" properly.
        This motivates the following definition:
        
        \definition     Suppose that $S$ is a set of $n$ pairwise disjoint oriented arcs each of which is
        a union of $K$ coherently oriented edges of a Heegaard diagram $D$, and suppose each arc
        in $S$  ``represents" the same word $W$. We will say that the members of $S$ are  ``parallel" if
        there exist two members of $S$, $L$ and $R$ such that for $1 \leq i \leq K$, there exists a
        complementary region $N_i$ of the 2-gon formed by the edges $L_i$ and $R_i$ and the vertices
        joined by $L_i$ and $R_i$, such that the interior of Ni lies to the  ``right" of the oriented
        edge $L_i$, $N_i$ contains no vertices of $D$, and the $i$th edge of each member of $S$ lies in 
        the closure of $N_i$. 
        
                The next lemma essentially shows that if the members of a set of edges are parallel
        in a realization of a minimal length presentation, then the members of the  ``next" set of
        edges are also parallel and the successive sets of parallel edges  ``match up".
        
        \lemma  Suppose that $S$ is a set of $n$ pairwise disjoint oriented arcs each of which is
        a union of two coherently oriented edges of a Heegaard diagram $D$, each arc in S
         ``represents" the same word $W$, $D$  ``realizes" a minimal length presentation $P$, and the
        first edges of the arcs in $S$ are all parallel. Then the members of $S$ are parallel.
        
        \pf   Consider the word $W$, $W$ has one of four forms: $XYZ, XYY, XXY$ or $XXX$, where
        $X$, $Y$ and $Z$ are distinct letters. By drawing representative diagrams for each of these
        cases, and making use of the fact that a Heegaard diagram $D$, with $|D| = |P|$, which
        realizes a minimal length presentation $P$, has no cut vertices, one can easily show
        that the desired conclusion holds.
        
                The next corollary follows immediately by induction on $K$.
        
        \corollary      Suppose that $S$ is a set of $n$ pairwise disjoint oriented arcs each of which is
        a union of $K$ coherently oriented edges of a Heegaard diagram $D$, each arc in $S$
         ``represents" the same word $W$, $D$  ``realizes" a minimal length presentation $P$, $|D| = |P|$,
        and the first edges of the arcs in $S$ are all parallel. Then the members of $S$ are parallel.   
        
        
                                                                                \section{ANNULI}
                        
                Under certain conditions, the program will assert that an  ``annulus" exists in a
        diagram. We can now use the preceding results to explain and justify the program's
        claims.
        
                Suppose that $P$ is a minimal length presentation, $D$ is a Heegaard diagram which
        realizes $P$, $|D| = |P|$, the pair $\{ X, Y\}$ is a pair of separating vertices of the reduced
        Whitehead graph $G$ of $P$, and $C$ is a component of the separation obtained by deleting
        vertex $X$ and vertex $Y$ from $G$. Suppose that each  ``path" which leaves component $C$ via
        vertex $X$ returns to $C$ via vertex $Y$ and in addition each such  ``path" represents the same
        word $W$. Then the program will assert that an annulus $A$ exists in $D$ such that $A$  ``swallows"
        $C$ and otherwise $A$ follows the paths that leave $C$ via $X$ and return to $C$ via $Y$.
        
                We can apply the preceding results, by observing that since the paths in question
        only have their endpoints in $C$, we can collapse $C$ to a point, and treat it as a
         ``vertex" of $D$.        Then consider these paths as tracing out realizations of the word $cWC$,
        and apply the above results. We see that these paths must all be parallel. Hence the
        required annulus exists.
        
                Note that this yields an essential annulus $A$, in the underlying handlebody $H$, since
        there exists a vertex $V$ of $D$, distinct from $X$ and $Y$, such that $V$ does not lie in $C$.
        Thus $A$ is not parallel into the boundary of $H$, and $A$ is, of course, incompressible in $H$.
        

        
\end{document}
        
                                                        
